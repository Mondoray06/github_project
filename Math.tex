
\documentclass[3pt]{article}
%%%%%%%%%%%%%%%%%%%%%%%%%%%%%%%%%%%%%%%%%%%%%%%%%%%%%%%%%%%%%%%%%%%%%%%%%%%%%%%%%%%%%%%%%%%%%%%%%%%%%%%%%%%%%%%%%%%%%%%%%%%%
\usepackage{amssymb}
\usepackage{amsfonts}
\usepackage{amsmath}
\usepackage{fancyhdr}
\usepackage{geometry}
\usepackage{amscd}

\setcounter{MaxMatrixCols}{10}
%TCIDATA{OutputFilter=LATEX.DLL}
%TCIDATA{Version=4.10.0.2347}
%TCIDATA{Created=Wednesday, March 19, 2008 18:21:16}
%TCIDATA{LastRevised=Tuesday, November 19, 2019 18:02:22}
%TCIDATA{<META NAME="GraphicsSave" CONTENT="32">}
%TCIDATA{<META NAME="DocumentShell" CONTENT="Articles\SW\JEEP -  A General Purpose Vehicle">}
%TCIDATA{Language=American English}
%TCIDATA{CSTFile=LaTeX article (bright).cst}

\geometry{ hmargin=1.5cm, vmargin=1.5cm }
\newtheorem{theorem}{Theorem}
\newtheorem{acknowledgement}[theorem]{Acknowledgement}
\newtheorem{algorithm}[theorem]{Algorithm}
\newtheorem{axiom}[theorem]{Axiom}
\newtheorem{case}[theorem]{Case}
\newtheorem{claim}[theorem]{Claim}
\newtheorem{conclusion}[theorem]{Conclusion}
\newtheorem{condition}[theorem]{Condition}
\newtheorem{conjecture}[theorem]{Conjecture}
\newtheorem{corollary}[theorem]{Corollary}
\newtheorem{criterion}[theorem]{Criterion}
\newtheorem{definition}[theorem]{D\'{e}finition}
\newtheorem{example}[theorem]{Example}
\newtheorem{exercise}[theorem]{Exercise}
\newtheorem{lemma}[theorem]{Lemme}
\newtheorem{notation}[theorem]{Notation}
\newtheorem{problem}[theorem]{Problem}
\newtheorem{proposition}[theorem]{Proposition}
\newtheorem{remark}[theorem]{Remarque}
\newtheorem{solution}[theorem]{Solution}
\newtheorem{summary}[theorem]{Summary}
\newenvironment{Proof}[1][Proof]{\noindent\textbf{Proof} }{\ \rule{0.5em}{0.5em}}
\input{tcilatex}

\begin{document}

\title{Math}
\author{T.\ Monedero \\
%EndAName
Natixis Fixed Income Department:\\
quantitative analysis }
\maketitle

\begin{abstract}
\end{abstract}

\tableofcontents

\bigskip

\bigskip

\bigskip

\bigskip

\bigskip

\bigskip

\bigskip

\bigskip

\bigskip

\bigskip

\bigskip

\bigskip

\bigskip

\bigskip

\bigskip

\bigskip

\bigskip

\bigskip

\section{Mesure et Integration}

\bigskip

\subsection{Mesure}

\subsubsection{Espace Mesurable}

\bigskip

\begin{definition}[Classes Monotones]
Soit $X$ un sous-ensemble $\mathcal{N\subset }\mathcal{P}(X)$ appel\'{e} une
classe monotone si :

\begin{itemize}
\item $X\in \mathcal{N}$.

\item Si $A,B\in \mathcal{N}$, et $A\subset B$ alors $B$ $\backslash $ $A$ $%
\in \mathcal{N}$

\item Si $A_{n}\in \mathcal{N}$, $\forall n\in \mathbb{N}$, et que $%
A_{n}\subset A_{n+1}$\ alors $\tbigcup\limits_{n\in \mathbb{N}}A_{n}\in 
\mathcal{N}$.
\end{itemize}
\end{definition}

\bigskip

\begin{remark}
\begin{itemize}
\item Si $A\in \mathcal{N}$, alors $A^{C}=X$ $\backslash $ $A\in \mathcal{N}$

\item Toute tribu est une classe monotone

\item Une classe monotone est une tribu ssi elle est stable par intersection
finie.

\item Toute intersection de classe monotone est encore une classe monotone.
Si $F$ est une famille de parties de $X$, on peut d\'{e}finir 
\begin{equation*}
\mathcal{N}(F)=\underset{\text{classe monotone sur }X,\text{ }F\subset 
\mathcal{N}}{\bigcap \mathcal{N}}
\end{equation*}%
Alors, $\mathcal{N}(F)$ est une classe monotone sur $X$ appel\'{e}e la
classe monotone engendr\'{e}e par $F.$ C'est la petite classe monotone sur $%
X $ qui contient $F$.
\end{itemize}
\end{remark}

\bigskip

\begin{definition}
Soit $X$ un ensemble. On appelle tribu ou $\sigma $- alg\`{e}bre sur $X$ une
famille $\mathcal{M}$ de partie de $X$ poss\'{e}dant les propri\'{e}t\'{e}s
suivantes :

\begin{itemize}
\item $X\in \mathcal{M}$.

\item Si $A\in \mathcal{M}$, alors $A^{C}\in \mathcal{M}$ ( ou $A^{C}=X$ $%
\backslash $ $A$ est le compl\'{e}mentaire de $A$ dans $X$ ).

\item Si $A_{n}\in \mathcal{M}$, $\forall n\in \mathbb{N}$, alors $%
\tbigcup\limits_{n\in \mathbb{N}}A_{n}\in \mathcal{M}$.

Les \'{e}l\'{e}ments de $\mathcal{M}$ sont appel\'{e}s les parties
mesurables de $X$. On dit que ($X,\mathcal{M}$) est un espace mesurable.
\end{itemize}
\end{definition}

\bigskip

$\mathcal{M=}\left\{ \varnothing ,X\right\} $ est la plus petite tribu de $X$
et $\mathcal{M=}\mathcal{P}(X)$ la plus grande. De plus, $\mathcal{M}$ est
stable par union ou intersection finie. En effet, si $A_{n}\in \mathcal{M}$, 
$\forall n\in \mathbb{N}$, alors $\tbigcap\limits_{n\in \mathbb{N}}A_{n}\in 
\mathcal{M}$ car $\left( \tbigcap\limits_{n\in \mathbb{N}}A_{n}\right)
^{C}=\tbigcup\limits_{n\in \mathbb{N}}A_{n}^{C}$. Enfin, si $A$ et $B$ sont
mesurables, alors la diff\'{e}rence non sym\'{e}trique $A$ $\backslash $ $%
B=A\tbigcap B^{C}\in \mathcal{M}$.

\bigskip

\begin{lemma}
Soit $\left\{ \mathcal{M}_{i}\right\} _{i\in I}$ une famille quelconque de
tribus sur $X$. Alors $\mathcal{M=}\tbigcap_{i\in I}\mathcal{M}_{i}$ est
encore une tribu sur $X$.
\end{lemma}

\bigskip

\begin{definition}
Soit $F$ une famille de parties de $X\ $et $\left\{ \mathcal{M}%
_{i}^{F}\right\} _{i\in I}$ la famille de tribus sur $X$ contenant $F$ ( i.e 
$\forall i\in I$, $F\subset \mathcal{M}_{i}^{F}$ ). On note 
\begin{equation*}
\sigma (F)=\tbigcap_{i\in I}\mathcal{M}_{i}^{F}
\end{equation*}%
la tribu engendr\'{e}e par $F$ sur $X$. C'est le plus petite tribu sur $X$
qui contient $F$.
\end{definition}

\bigskip

\begin{lemma}
Si $F\subset \mathcal{P}(X)$ est une famille de partie de$\ X$ stable par
intersections finies alors $\mathcal{N}(F)=\mathcal{\sigma }(F).$
\end{lemma}

\bigskip

\begin{corollary}
Soit $(X,\mathcal{M})$ un espace mesurable muni de deux mesures $\mu $\ et$\
\upsilon $. Supposons qu'il existe une famille $F$ de parties de $\mathcal{M}
$\ telle que

\begin{itemize}
\item $F$ est stable par intersection finie et $\sigma (F)=\mathcal{M}$

\item $\mu (A)=\upsilon (A),$ $\forall A\in F$
\end{itemize}

On suppose en outre que

\begin{itemize}
\item soit que $\mu (X)=\upsilon (X)<\infty $

\item soit qu'il existeune famille $\left\{ E_{n}\right\} _{n\in \mathbb{N}}$%
\ d'\'{e}l\'{e}ments de $F,$ telle que $E_{n}\subset E_{n+1},$ $%
\bigcup_{n\in \mathbb{N}}E_{n}=X$ et $\mu (E_{n})=\upsilon (E_{n})<\infty $
pour tout $n\in \mathbb{N}$ alors%
\begin{equation*}
\mu =\upsilon ,\text{ ie }\mu (A)=\upsilon (A)\text{, }\forall A\in \mathcal{%
M}
\end{equation*}
\end{itemize}
\end{corollary}

\bigskip

\begin{example}[Unicit\'{e} de la mesure de Lebesgue]
On prend $X=\mathbb{R}^{d}$, $\mathcal{M}=\mathcal{B(}\mathbb{R}^{d}\mathcal{%
)}$, $F$ la famille des pav\'{e}s ouverts et $E_{n}=\left] -n,n\right[ ^{d}$%
. En appliquan le b) du corrolaire ci dessus, on voit qu'une mesure
borelienne sur $\mathbb{R}^{d}$ finie sur les born\'{e}s est entierement d%
\'{e}termin\'{e}e par ses valeurs sur les pav\'{e}s ouverts. Ceci montre
donc l'unicit\'{e} de la mesure de Lebesgue sur $\mathbb{R}^{d}$.
\end{example}

\bigskip

\begin{definition}
Unt topologie sur $X\ $est une famille $\mathcal{T}$ de parties de $X$
telles que :

\begin{itemize}
\item $\varnothing \in \mathcal{T},$ $X\in \mathcal{T}$.

\item Si $O_{1},...,O_{n}\in \mathcal{T},$ alors $\tbigcap%
\limits_{i=1}^{n}O_{i}\in \mathcal{T}$.

\item Si $\left\{ O_{i}\right\} _{i\in I}$\ est une famille quelconque d'%
\'{e}l\'{e}ments de $\mathcal{T}$ alors $\tbigcup_{i\in I}O_{i}\in \mathcal{T%
}$.

Les \'{e}l\'{e}ments de $\mathcal{T}$ s'appelent les ouverts de $X$. On dit
que ( $X$, $\mathcal{T}$ ) est un espace topologique
\end{itemize}
\end{definition}

\bigskip

\begin{definition}
Soit ( $X$, $\mathcal{T}$ ) un espace topologique. On appelle tribu de Borel
sur $X$ la tribu engendr\'{e}e par les ouverts de $X$ : $\mathcal{M}=\sigma (%
\mathcal{T})$ . La tribu $\mathcal{B(}\mathbb{R}\mathcal{)}$ est engendr\'{e}%
e par les intervalles $\left] a,+\infty \right[ $ pour $a\in \mathbb{R}.$
\end{definition}

\bigskip \bigskip

\bigskip

\bigskip

\bigskip

\subsubsection{Mesure Positive}

\bigskip

\begin{definition}[Mesure Exterieure]
Soit $X$ un esemble quelconque. On appelle mesure ext\'{e}rieur sur $X$ une
application \ $\mu ^{\ast }:\mathcal{P(}\mathbb{X}\mathcal{)\rightarrow }%
\left[ 0,+\infty \right] $ telle que

\begin{itemize}
\item $\mu ^{\ast }(\varnothing )=0$

\item $\mu ^{\ast }\ $est croissante : $\mu ^{\ast }(A)=\mu ^{\ast }(B)$ si $%
A\subset B$

\item $\mu ^{\ast }\ $est sous additive : si $\left\{ A_{n}\right\} _{n}$
est une famille de parties de $X$ alors : $\mu ^{\ast }\left(
\tbigcup\limits_{n\in \mathbb{N}}A_{n}\right) \leq \tsum\limits_{n\in 
\mathbb{N}}\mu ^{\ast }\left( A_{n}\right) $
\end{itemize}
\end{definition}

\bigskip

\begin{definition}[Regularit\'{e}]
Soit $X$ un ensemble muni d'une mesure ext\'{e}rieure $\mu ^{\ast }$. On dit
qu'une partie $B\subset X$ est $\mu ^{\ast }$-r\'{e}guli\`{e}re si pour
toutes parties $A$ de $X$ on a 
\begin{equation*}
\mu ^{\ast }\left( A\right) =\mu ^{\ast }\left( A\cap B\right) +\mu ^{\ast
}\left( A\cap B^{C}\right)
\end{equation*}%
On note $\mathcal{M(}\mu ^{\ast }\mathcal{)}$ l'ensemble des parties $\mu
^{\ast }$-r\'{e}guli\`{e}re de X.
\end{definition}

\bigskip

\begin{definition}
Soit ($X,\mathcal{M}$) un espace mesurable. On appelle mesure positive sur $%
X $ une application $\mu :\mathcal{M\rightarrow }\left[ 0,+\infty \right[ $
verifiant :

\begin{itemize}
\item $\mu (\varnothing )=0$

\item Additivit\'{e} d\'{e}nonbrable : si $\left\{ A_{n}\right\} _{n\in 
\mathbb{N}}$ est une famille d\'{e}nombrable d'ensembles mesurables deux a
deux disjoints alors%
\begin{equation*}
\mu \left( \tbigcup_{n\in \mathbb{N}}A_{n}\right) =\tsum_{n\in \mathbb{N}%
}\mu \left( A_{n}\right)
\end{equation*}%
On dit que ($X,\mathcal{M},\mu $) est un espace mesur\'{e}.
\end{itemize}
\end{definition}

\bigskip

\begin{proposition}
Une mesure positive poss\`{e}de les propri\`{e}t\'{e}s suivantes :

\begin{itemize}
\item Si $A,B\in \mathcal{M}$ et $A\subset B$, alors $\mu (A)\leqslant \mu
\left( B\right) $\ ( Monotonie ).

\item Si $A_{n}\in \mathcal{M}$, $\forall n\in \mathbb{N}$ alors $\mu \left(
\tbigcup_{n\in \mathbb{N}}A_{n}\right) \leqslant \tsum_{n\in \mathbb{N}}\mu
\left( A_{n}\right) $ ( Sous additivit\'{e} ).

\item Si $A_{n}\in \mathcal{M}$, $\forall n\in \mathbb{N}$ et si $%
A_{n}\subset A_{n+1},\forall n\in \mathbb{N}$ alors $\mu \left(
\tbigcup_{n\in \mathbb{N}}A_{n}\right) =\underset{n\rightarrow +\infty }{%
\lim }\mu \left( A_{n}\right) .$

\item Si $A_{n}\in \mathcal{M}$, $\forall n\in \mathbb{N}$ et si $%
A_{n}\supset A_{n+1},\forall n\in \mathbb{N}$ avec $\mu \left( A_{0}\right)
<\infty $ alors $\mu \left( \tbigcap_{n\in \mathbb{N}}A_{n}\right) =\underset%
{n\rightarrow +\infty }{\lim }\mu \left( A_{n}\right) .$
\end{itemize}
\end{proposition}

\bigskip

\begin{proposition}
$\mathcal{M(}\mu ^{\ast }\mathcal{)}$ est une tribu sur $X$ contenant toutes
les parties $B\subset X$ telles que $\mu ^{\ast }\left( B\right) =0$\ et la
restriction de $\mu ^{\ast }$ \`{a} $\mathcal{M(}\mu ^{\ast }\mathcal{)}$
est une mesure.
\end{proposition}

\subsubsection{Completion de Mesure}

\bigskip

\begin{definition}
Soit ($X,\mathcal{M},\mu $) est un espace mesur\'{e}. On dit que

\begin{itemize}
\item $A\subset X$ est n\'{e}gligeable pour la mesure $\mu $ si $A\in 
\mathcal{M}$ et $\mu (A)=0.$

\item La mesure $\mu $ est compl\`{e}te si tout sous ensemble d'un ensemble n%
\'{e}glig\'{e}able est encore n\'{e}glig\'{e}able.
\end{itemize}
\end{definition}

\bigskip

\begin{proposition}
Soit ($X,\mathcal{M},\mu $) est un espace mesur\'{e}. Soit $\mathcal{M}%
^{\ast }$ l'ensemble de toutes les parties $E$ de $X$ telles qu'il existe $%
A,B\in \mathcal{M}$ avec $A\subset E\subset B$ et $\mu (B$ $\backslash $ $%
A)=0.$ On d\'{e}finit alors $\mu ^{\ast }(E)=\mu (A).$ Ainsi, $\mathcal{M}%
^{\ast }$ est une tribu sur $X$ et $\mu ^{\ast }$ une mesure compl\`{e}te $%
\mathcal{M}$ sur qui prolonge $\mu $.
\end{proposition}

\subsubsection{Mesure de Lebesgue}

\bigskip

\begin{theorem}
Il existe une unique mesure positive sur $(\mathbb{R},\mathcal{B(\mathbb{R})}%
)$, not\'{e}e $\lambda $, telle que 
\begin{equation*}
\lambda (\left] a,b\right[ )=b-a,\forall a,b\in \mathbb{R}\times \mathbb{R}%
\text{ }\backslash \text{ }a<b
\end{equation*}%
$\lambda $ est appell\'{e}e mesure de Lebesgue sur $\mathbb{R}$. La mesure
de Lebesgue est diffuse : $\forall x\in \mathbb{R}$, $\lambda (\left\{
x\right\} )=0$. Par cons\'{e}quent,%
\begin{equation*}
\lambda (\left] a,b\right[ )=\lambda (\left[ a,b\right[ )=\lambda (\left] a,b%
\right] )=\lambda (\left[ a,b\right] )=b-a,\text{ }a\leq b
\end{equation*}
\end{theorem}

\bigskip

\begin{definition}
On appelle tribu de Lebesque sur $\mathbb{R}$ , et on note $\mathcal{L(}%
\mathbb{R}\mathcal{)}$, la tribu qui compl\`{e}te la tribu de Borel $%
\mathcal{B(}\mathbb{R}\mathcal{)}$ pour la mesure de Lebesgue $\lambda .$ On
appelle encore mesure de Lebesgue la mesure compl\'{e}t\'{e}e $\lambda $: $%
\mathcal{L(}\mathbb{R}\mathcal{)\rightarrow }\left[ 0,+\infty \right] .$
\end{definition}

\bigskip

\begin{definition}
Un pav\'{e} $P$ de $\mathcal{\mathbb{R}}^{d}$ est un produit \ dintervalles
born\'{e}s $P=I_{1}\times I_{2}\times ...\times I_{d},$ $I_{j}\subset 
\mathbb{R}$ intervalle born\'{e}. La mesure du pav\'{e} $P$ est not\'{e}e 
\begin{equation*}
mes(P)=l(I_{1})\cdot l(I_{2})\cdot ...\cdot l(I_{d})
\end{equation*}%
ou $l(I_{j})$ est la longueur du segment $I_{j}.$Pour toute partie $A$ de $%
\mathbb{R}^{d}$, on d\'{e}finit 
\begin{equation*}
\lambda ^{\ast }(A)=\inf \left\{ \tsum\limits_{i\in \mathbb{N}}mes(P_{i})%
\text{ }|\text{ }A\subset \tbigcup\limits_{i\in \mathbb{N}}P_{i},\text{ }%
P_{i}\text{ pav\'{e} ouvert de }\mathbb{R}^{d}\right\}
\end{equation*}%
L'infimum est pris sur tous les recouvrements d\'{e}nombrables de $A$ par
des pav\'{e}es ouverts.
\end{definition}

\bigskip

\begin{theorem}
On a les assertions suivantes :

\begin{itemize}
\item $\mathcal{\lambda ^{\ast }}$ est une mesure exterieue sur $\mathcal{%
\mathbb{R}}^{d}$

\item La tribu $\mathcal{M(\lambda ^{\ast })}$\ contient la tribu de Borel $%
\mathcal{B(\mathbb{R}}^{d}\mathcal{)}$

\item $\lambda ^{\ast }(P)=mes(P)$, pour tout pav\'{e} $P\subset \mathcal{%
\mathbb{R}}^{d}$
\end{itemize}
\end{theorem}

\bigskip

\begin{definition}
On appelle mesure de Lebesgue sur $\mathcal{\mathbb{R}}^{d}$\ la
restriction, not\'{e}e $\lambda $, de la mesure exterieure $\mathcal{\lambda
^{\ast }}$ \`{a} $\mathcal{B(\mathbb{R}}^{d}\mathcal{)}$ ou \`{a} $\mathcal{%
M(\lambda ^{\ast })}$.
\end{definition}

\bigskip

\begin{lemma}
Si $P,P_{1},...,P_{N}$ sont des pav\'{e}s de $\mathbb{R}^{d}$\ avec $%
P\subset $ $\bigcup_{i=1}^{N}Pi$ alors%
\begin{equation*}
mes(P)\leq \sum_{i=1}^{N}mes(Pi)
\end{equation*}
\end{lemma}

\bigskip

\begin{theorem}
Il existe une unique mesure positive sur $(\mathbb{R}^{d},\mathcal{B(\mathbb{%
R}}^{d}\mathcal{)})$, not\'{e}e $\lambda $, telle que pour tout pav\'{e} $P=%
\left[ a_{1},b_{1}\right] \times \left[ a_{2},b_{2}\right] \times ...\times %
\left[ a_{d},b_{d}\right] \subset \mathcal{\mathbb{R}}^{d},$ on ait 
\begin{equation*}
\lambda (P)=\tprod\limits_{i=1}^{d}(b_{i}-a_{i})
\end{equation*}%
Comme pr\'{e}cedemment, on peut compl\'{e}ter la tribu $\mathcal{B(\mathbb{R}%
}^{d}\mathcal{)}$ et \'{e}tendre la mesure $\lambda $ \`{a} $\mathcal{L(%
\mathbb{R}}^{d}\mathcal{)}$. La mesure de Lebesgue ( sur $\mathcal{B(\mathbb{%
R}}^{d}\mathcal{)}$\ ou $\mathcal{L(\mathbb{R}}^{d}\mathcal{)}$\ ) poss\`{e}%
de les priopri\`{e}t\'{e}s suivantes :

\begin{itemize}
\item $\lambda $ est invariante par translation et rotation

\item $\lambda $ est r\'{e}guli\`{e}re, i.e $\forall E\subset \mathcal{%
\mathbb{R}}^{d}$ mesurable on a : \newline

\begin{itemize}
\item $\lambda (P)=\sup \left\{ \lambda (K)\text{ }|\text{\ }K\text{
compact, }K\subset E\right\} $ ( regularit\'{e} int\'{e}rieure ).

\item $\lambda (P)=\inf \left\{ \lambda (V)\text{ }|\text{\ }V\text{ ouvert, 
}V\supset E\right\} $ ( regularit\'{e} exterieure ).
\end{itemize}
\end{itemize}
\end{theorem}

\bigskip

On sait que $\lambda $ se prolonge en une mesure sur $\mathcal{L(}\mathbb{R}%
^{d}\mathcal{)}$ avec $\mathcal{L(}\mathbb{R}^{d}\mathcal{)=\sigma (\mathcal{%
B(}}\mathbb{R}^{d}\mathcal{\mathcal{)}},N\mathcal{)}$ ou $N=\left\{ A\subset 
\mathcal{\mathbb{R}}^{d}\text{ }|\text{\ }\exists B\in \mathcal{\mathcal{B(}}%
\mathbb{R}^{d}\mathcal{\mathcal{)}}\text{ avec }A\subset b\text{\ et }%
\lambda (B)=0\right\} $ est l'ensemble des parties n\'{e}glig\'{e}ables.

On sait aussi que $\lambda $ se prolonge a la tribu $\mathcal{M(}\lambda
^{\ast })\supset \mathcal{\mathcal{B(}}\mathbb{R}^{d}\mathcal{\mathcal{)}}$
avec $\mathcal{M(}\lambda ^{\ast })=\left\{ B\subset \mathbb{R}^{d}\text{ }|%
\text{ }\lambda ^{\ast }(A)=\lambda ^{\ast }(A\cap B)+\lambda ^{\ast }(A\cap
B^{C}),\text{ }\forall A\subset \mathbb{R}^{d}\right\} .$ On peut se
demander si $\mathcal{M(}\lambda ^{\ast })$ est beaucoup plus grande que la
tribu $\mathcal{B(\mathbb{R}}^{d}\mathcal{)}$\ et si elle \`{a} un lien avec
la tribu compl\'{e}t\'{e}e $\mathcal{L(\mathbb{R}}^{d}\mathcal{)}$.

\bigskip

\begin{proposition}
\begin{itemize}
\item On a $\mathcal{M(}\lambda ^{\ast })=\mathcal{L(\mathbb{R}}^{d}\mathcal{%
)}$.

\item Soit $\mathcal{M=\mathcal{B(}}\mathbb{R}^{d}\mathcal{\mathcal{)}}$ ou $%
\mathcal{L(}\mathbb{R}^{d}\mathcal{)}$. La mesure de Lebesgue est invariante
par translation au sens ou pour tout $A\in \mathcal{M}$ et $x\in \mathbb{R}%
^{d},$ on a $x+A\in \mathcal{M}$ et $\lambda (x+A)=\lambda (A).$

\item Si $\mu :\mathcal{\mathcal{B(}}\mathbb{R}^{d}\mathcal{\mathcal{)}%
\rightarrow }\left[ 0,+\infty \right] $ est une mesure invaiante par
translation et finie sur les born\'{e}s alors il existe une constante $c\geq
0$ telle que $\mu =c\lambda $
\end{itemize}
\end{proposition}

\begin{theorem}
\begin{itemize}
\item $\forall A\subset \mathcal{L(}\mathbb{R}^{d}\mathcal{)}$ on a : 
\newline

\begin{itemize}
\item $\lambda (A)=\sup \left\{ \lambda (K)\text{ }|\text{\ }K\text{
compact, }K\subset A\right\} $ ( regularit\'{e} int\'{e}rieure ).

\item $\lambda (A)=\inf \left\{ \lambda (U)\text{ }|\text{\ }U\text{ ouvert, 
}U\supset A\right\} $ ( regularit\'{e} exterieure ).
\end{itemize}
\end{itemize}
\end{theorem}

\bigskip

\subsubsection{Repr\'{e}sentation de Riez et comparaison avec l'int\'{e}%
grale de Rieman}

\bigskip

\subsection{Th\'{e}orie de l'int\'{e}gration}

\bigskip

\subsubsection{Fonction Mesurable}

\bigskip

\paragraph{Definitions}

\bigskip

\begin{definition}
Soit $(X,\mathcal{M},)$ et $(Y,\mathcal{N})$ deux espaces mesurables. On dit
qu'une application $f:X\rightarrow Y$ est mesurable pour les tribus $%
\mathcal{M}$ et $\mathcal{N}$\ si 
\begin{equation*}
f^{-1}(B)\in \mathcal{M}\text{, }\forall \text{ }B\in \mathcal{N}
\end{equation*}
\end{definition}

\bigskip

\begin{remark}
Etant donn\'{e}e deux espaces topologiques $(X,\mathcal{T})$ et $(Y,\mathcal{%
S})$, la definition d'application continue pour les topologies $\mathcal{T}$
et $\mathcal{S}$ est analogue \`{a} celle de mesurabilit\'{e}, i.e%
\begin{equation*}
f^{-1}(B)\in \mathcal{T}\text{, }\forall \text{ }B\in \mathcal{S}
\end{equation*}
\end{remark}

\bigskip

\begin{remark}
Si $Y$ est un ensemble quelconque, $\mathcal{N=}\left\{ \varnothing
,Y\right\} $ est la plus petite tribu sur $Y$ rendant $f$ mesurable. La
tribu image de $\mathcal{M}$ par $f\ $%
\begin{equation*}
\mathcal{N}^{\ast }\mathcal{=}\left\{ B\subset Y\text{ }|\text{ }%
f^{-1}(B)\in \mathcal{M}\right\}
\end{equation*}%
est la plus grande tribu sur $Y$ rendant $f$ mesurable.
\end{remark}

\bigskip

\begin{remark}
Si $X$ est un ensemble quelconque, $\mathcal{M=P}(X)$ est la plus grande
tribu sur $X$ rendant $f$ mesurable. La tribu engendr\'{e}e par $f\ $%
\begin{equation*}
\mathcal{M}^{\ast }\mathcal{=}\left\{ \text{ }f^{-1}(B)|\text{ }B\in 
\mathcal{N}\right\}
\end{equation*}%
est la plus petite tribu sur $X$ rendant $f$ mesurable.
\end{remark}

\paragraph{Stabilit\'{e}}

\bigskip

\begin{itemize}
\item Si $f:(X_{1},\mathcal{M}_{1})\rightarrow (X_{2},\mathcal{M}_{2})$ et $%
g:(X_{2},\mathcal{M}_{2})\rightarrow (X_{3},\mathcal{M}_{3})$ sont
mesurables, alors $g\circ f:(X_{1},\mathcal{M}_{1})\rightarrow (X_{3},%
\mathcal{M}_{3})$ est mesurable.

\item Soient $(X,\mathcal{M})$ un espace mesurable, $(Y,T)$ un espace
topologique, $f_{1},f_{2}:X\rightarrow \mathbb{R}$ des applications
mesurables et $\Phi :\mathbb{R}^{2}\rightarrow Y$ une application continue.
Alors, $h:X\rightarrow Y$ definie par $\forall x\in X,$ $h(x)=\Phi
(f_{1}(x),f_{2}(x))$ est mesurable.

\item Soient $f,g:X\rightarrow \mathbb{R}$ deux fonctions mesurables. Alors $%
f+g,$ $fg$, $\min (f,g)$, $\max (f,g)$ sont mesurables.

\item Si $f:X\rightarrow \mathbb{R}$ est mesurable,. alors $f_{+}=\max
(f,0), $ $f_{-}=\min (f,0)$ et $|f|$ $=f_{+}+f_{-}$ sont mesurables.

\item Si $f:X\rightarrow \mathbb{R}$ est mesurable et si $\forall x\in X,$ $%
f(x)\neq 0,$ alors $g$ d\'{e}finie par $g(x)=\frac{1}{f(x)}$ est mesurable.

\item Soit $\mathbb{\bar{R}}=\mathbb{R}+\{-\infty ,+\infty \}.$ Les ouverts
de $\mathbb{\bar{R}}$ sont les unions d'intervalles de la forme $\left[
-\infty ,a\right[ ,\left] a,b\right[ ,\left] b,+\infty \right] ,$ $\forall
a,b\in \mathbb{R}.$ Soit $(X,\mathcal{M})$ un espace mesurable et $%
\{f_{n}\}_{n}$ une suite\ de fonctions de $X$ dnas $\mathbb{\bar{R}}$. Alors 
$\underset{n}{\sup }f_{n},$ $\underset{n}{\inf }f_{n},$ $\underset{%
n\rightarrow \infty }{\lim \sup }f_{n},$ $\underset{n\rightarrow \infty }{%
\lim \inf }f_{n}:X\rightarrow \mathbb{\bar{R}}$ sont des applications
mesurables avec $\forall x\in X$%
\begin{eqnarray*}
\left( \underset{n}{\sup }f_{n}\right) (x) &=&\underset{n}{\sup }f_{n}(x) \\
\left( \underset{n\rightarrow \infty }{\lim \sup }f_{n}\right) (x) &=&%
\underset{n\rightarrow \infty \text{ }k\geq n}{\lim \sup }f_{k}(x)
\end{eqnarray*}
\end{itemize}

On definit de meme $\underset{n}{\inf }f_{n}$ et $\underset{n\rightarrow
\infty }{\lim \inf }f_{n}.$ En particulier, si $f(x)=\underset{n\rightarrow
\infty }{\lim }f_{n}(x)$ existe $\forall x\in X$ alors $f:X\rightarrow 
\mathbb{\bar{R}}$ est mesurable. Plus generalement, l'ensemble $\{x\in X$ $|$
$\underset{n\rightarrow \infty }{\lim }f_{n}(x)$ existe$\}$ est mesurable.

\bigskip

\subsubsection{Fonctions \'{e}tag\'{e}es}

\bigskip

\begin{definition}
Soit $(X,\mathcal{M})$ est un espace mesurable. On dit qu'une application
mesurable $f:X\rightarrow \mathbb{R}$ est \'{e}tag\'{e}e si $f$ ne prend
qu'une nombre fini de valeurs. Pour $i=1..n,$ on note $\alpha _{i}$ les
valeurs de $f$ et $A_{i}=f^{-1}(\alpha _{i})$, alors%
\begin{equation*}
f=\tsum\limits_{i=1}^{n}\alpha _{i}\mathbf{1}_{A_{i}}
\end{equation*}
\end{definition}

\bigskip

\begin{proposition}
Soit $f:(X,\mathcal{M})\rightarrow \left[ 0,+\infty \right] $ une fonction
mesurable. Alors il existe une suite croissante de fonctions mesurables \'{e}%
tag\'{e}ees qui converge ponctuellement vers $f$.
\end{proposition}

\bigskip

On suppose \`{a} present que $(X,\mathcal{M},\mu )$ est un espace mesur\'{e}.

\bigskip

\begin{definition}
On note $\varepsilon _{+}$ l'ensemble des fonctions mesurables \'{e}tag\'{e}%
es $f:(X,\mathcal{M},\mu )\rightarrow \left[ 0,+\infty \right[ $. On appelle
int\'{e}grale de $f$ pour la mesure $\mu $ l'application $I:$ $\varepsilon
_{+}\rightarrow \left[ 0,+\infty \right] $ d\'{e}finie par 
\begin{equation*}
\tint fd\mu =\tsum\limits_{i=1}^{n}\alpha _{i}\mu (A_{i})
\end{equation*}%
et poss\'{e}dant les propri\`{e}tes suivantes :

\begin{itemize}
\item $\tint (f+g)d\mu =\tint fd\mu +\tint gd\mu ,$ $\forall f,g\in
\varepsilon _{+}$ ( Additivit\'{e} ).

\item $\tint \lambda fd\mu =\lambda \tint fd\mu $ $\forall f\in \varepsilon
_{+},$ $\forall \lambda \in \mathbb{R}_{+}$ ( Homog\'{e}n\'{e}it\'{e} ).

\item Si $f$ et $g\in \varepsilon _{+}$ et si $f\leq g,$ alors $\tint fd\mu
\leq \tint gd\mu $ ( Monotonie ).
\end{itemize}
\end{definition}

\subsubsection{Integration Fonction Mesurable Positive}

\bigskip

\begin{definition}
Soit $f:X\rightarrow \left[ 0,+\infty \right] $ une fonction mesurable. On
appelle int\`{e}grale de $f$ sur $X$ pour la mesure $\mu $ la quantit\'{e}%
\begin{equation*}
\tint fd\mu =\sup \left\{ \tint hd\mu \text{ }|\text{ }h\in \varepsilon ,%
\text{ }h\leq f\right\} \in \left[ 0,+\infty \right]
\end{equation*}

Si $E\subset X$ est une partie mesurable, on note aussi $\tint_{E}fd\mu
=\tint f\mathbf{1}_{E}d\mu .$ Cette int\`{e}grale poss\`{e}de la propri\`{e}t%
\`{e} de monotonie.
\end{definition}

\bigskip

\begin{theorem}[Convergence monotone]
~\newline

Soit $f_{n}:X\rightarrow \left[ 0,+\infty \right] $ une suite croissante de
fonctions mesurables positives et soit $f$ $=\underset{n\rightarrow \infty }{%
\lim }f_{n}$ la limite ponctuelles des $f_{n}$. Alors $f$\ est mesurable et 
\begin{equation*}
\tint fd\mu =\underset{n\rightarrow \infty }{\lim }\tint f_{n}d\mu
\end{equation*}%
Cette int\`{e}grales poss\`{e}de les propri\`{e}t\'{e}s d'additivit\'{e} et
de monotonie.
\end{theorem}

\bigskip

\begin{definition}
Dans un espace mesur\'{e} $(X,\mathcal{M},\mu )$, on dit qu'une propro\`{e}t%
\'{e} $P(x)$, $x\in X$ est vrai presque partout ( ou $\mu $ presque partout)
si elle est vrai en dehors d'un ensemble n\'{e}glig\'{e}able ( $%
\Longleftrightarrow $ de mesure $\mu $\ nulle)
\end{definition}

\bigskip

\begin{proposition}
Soit $f:X\rightarrow \left[ 0,+\infty \right] $ une fonction mesurable.

\begin{itemize}
\item $\forall a>0,$ $\mu \left( \left\{ x\in X\text{ }|\text{ }f(x)\geq
a\right\} \right) \leq \frac{1}{a}\tint fd\mu $

\item $\tint fd\mu =0\Longleftrightarrow f=0$ $\mu $ presque partout

\item Si $\tint fd\mu <\infty ,$ alors $f<\infty $ $\mu $ presque partout

\item Si $f$ et $g:X\rightarrow $ $\left[ 0,+\infty \right] $ sont
mesurables alors $f=g$ $\mu $ presque partout $\Longrightarrow \tint fd\mu
=\tint gd\mu $
\end{itemize}
\end{proposition}

\bigskip

\begin{lemma}[de Fatou]
Soit $(f_{n}:X\rightarrow \left[ 0,+\infty \right] )_{n}$ une suite de
fonctions mesurables. Alors%
\begin{equation*}
\tint \left( \underset{n\rightarrow \infty }{\lim \inf }f_{n}\right) d\mu
\leq \underset{n\rightarrow \infty }{\lim \inf }\tint f_{n}d\mu
\end{equation*}
\end{lemma}

\bigskip

\begin{definition}[Mesure a densit\'{e}]
Soit $(X,\mathcal{M},\mu )$ un espace mesur\'{e}.et $f:X\rightarrow \left[
0,+\infty \right] $ une fonction mesurable. On d\'{e}finit une application $%
\upsilon :\mathcal{M}\rightarrow \left[ 0,+\infty \right] $ par 
\begin{equation*}
\upsilon (A)=\tint_{A}fd\mu =\tint f\mathbf{1}_{A}d\mu
\end{equation*}%
Alors $\upsilon $ est une mesure sur $(X,\mathcal{M})$ appel\'{e}e mesure de
densit\'{e} $f$ par rapport a $\mu .$ Si $A\in \mathcal{M}$ verifie que $\mu
(A)=0$\ alors $\upsilon (A)=0,$ on dit que $\upsilon $ est absolument
continue par rapport a $\mu .$
\end{definition}

\bigskip

\begin{definition}[Int\'{e}grabilit\'{e} sur $\mathbb{R}$]
Soit $(X,\mathcal{M},\mu )$ un espace mesur\'{e} quelconque et $%
f:X\rightarrow \mathbb{R}$ une fonction mesurable. On dit que $f$ est int%
\'{e}grable par rapport a si $\mu $ si $\tint |f|d\mu <\infty .$ Dans ce
cas, on pose 
\begin{equation*}
\tint fd\mu =\tint f_{+}d\mu +\tint f_{-}d\mu
\end{equation*}%
On note $\mathcal{L}^{1}(X,\mathcal{M},\mu )$ l'espace des fonctions int\'{e}%
grables sur $X$.
\end{definition}

\bigskip

\begin{remark}
Comme $f_{+},f_{-}\leq |f|$ alors les int\'{e}grales de $f_{+}$ et $f_{-}$
sont finies et la d\'{e}composition \`{a} du sens.
\end{remark}

\bigskip

\begin{proposition}
\begin{itemize}
\item $\mathcal{L}^{1}(X,\mathcal{M},\mu )$ est un espace vectoriel sur $%
\mathbb{R}$ et l'application $f\rightarrow \tint fd\mu $ est lin\'{e}aire

\item $\left\vert \tint fd\mu \right\vert \leq \tint \left\vert f\right\vert
d\mu ,$ $\forall f\in \mathcal{L}^{1}(X,\mathcal{M},\mu )$

\item Si $f,g\in \mathcal{L}^{1}(X,\mathcal{M},\mu )$ et si $f=g$ $\mu $
presque partout alors $\tint fd\mu $ $=\tint gd\mu $

\item Si $f,g\in \mathcal{L}^{1}(X,\mathcal{M},\mu )$ et $f$ $\leq g$\ alors 
$\tint fd\mu $ $\leq \tint gd\mu $
\end{itemize}
\end{proposition}

\bigskip

\begin{definition}
\begin{remark}
Il est possible d'\'{e}tendre la d\'{e}finition d'int\'{e}grabilit\'{e} et
ces propri\'{e}t\'{e}s ( hormis la derni\`{e}re ) sur l'ensemble $\mathbb{C}$%
. Dans ce cas $\left\vert \cdot \right\vert $ est le module et on pose 
\begin{equation*}
\tint fd\mu =\tint \mathfrak{R}(f)d\mu +\tint \mathfrak{I}(f)d\mu
\end{equation*}%
On note $\mathcal{L}_{\mathbb{C}}^{1}(X,\mathcal{M},\mu )$ l'espace des
fonctions int\'{e}grables sur $X$
\end{remark}
\end{definition}

\bigskip

\begin{theorem}[de la convergence domin\'{e}e]
Soit $(X,\mathcal{M},\mu )$ est un espace mesur\'{e} et $f_{n}:X\rightarrow 
\mathbb{C}$ une suite de fonctions mesurables. On suppose que :

\begin{itemize}
\item La limite $f(x)=\underset{n\rightarrow \infty }{\lim }f_{n}(x)$ existe 
$\forall x\in X$

\item Il existe $g:X\rightarrow \left[ 0,+\infty \right[ $ int\'{e}grable
telle que $\left\vert f_{n}(x)\right\vert \leq g(x),$ $\forall n\in \mathbb{N%
},$ $\forall x\in X$
\end{itemize}

Alors $f:X\rightarrow \mathbb{C}$ est int\'{e}grable et on a :%
\begin{equation*}
\tint fd\mu =\underset{n\rightarrow \infty }{\lim }\tint f_{n}d\mu \text{ et 
}\underset{n\rightarrow \infty }{\lim }\tint \left\vert f_{n}-f\right\vert
d\mu =0
\end{equation*}
\end{theorem}

\bigskip

\begin{remark}
Il est possible de relaxer l'hypoth\`{e}se $\forall x\in X$ par pour $\mu $
presque tout $x\in X.$
\end{remark}

\bigskip

\begin{corollary}
Soit $f_{n}:X\rightarrow \mathbb{C}$ une suite de fonctions int\'{e}grables
telles que $\tsum\limits_{n\in \mathbb{N}}\tint f_{n}d\mu <\infty .$ Alors
la s\'{e}rie $\tsum\limits_{n\in \mathbb{N}}f_{n}(x)$\ converge absolument
pour $\mu $ presque tout $x\in X$ vers une fonction $f$ int\'{e}grable et on
a 
\begin{equation*}
\tint fd\mu =\tsum\limits_{n\in \mathbb{N}}\tint f_{n}d\mu
\end{equation*}
\end{corollary}

\bigskip

\subsubsection{Integrale d\'{e}pendant d'un param\`{e}tre}

Soit $(X,\mathcal{M},\mu )$ est un espace mesur\'{e} et soit $(\Lambda ,d)$
un espace m\'{e}trique ( a d\'{e}finir). On consid\`{e}re une fonction 
\begin{eqnarray*}
f &:&X\times \Lambda \rightarrow \mathbb{C} \\
(x,\lambda ) &\longmapsto &f(x,\lambda )
\end{eqnarray*}%
int\'{e}grable sur $X$ pour la mesure $\mu .$ On peut donc d\'{e}finir la
fonction $F:\Lambda \rightarrow \mathbb{C}$ par%
\begin{equation*}
F(\lambda )=\tint_{X}f(x,\lambda )d\mu _{x}\equiv \tint_{X}f(x,\lambda )dx%
\text{ }\forall \lambda \in \Lambda
\end{equation*}%
Afin d'harmoniser les notations, on notera pour une fonction $g:X\rightarrow 
\mathbb{C}$ de facon equivalente 
\begin{equation*}
\tint gd\mu =\tint g(x)d\mu _{x}=\tint g(x)dx
\end{equation*}

\bigskip

\begin{theorem}[de continuit\'{e} ]
On suppose

\begin{itemize}
\item $\forall \lambda \in \Lambda ,$ la fonction $x\longmapsto f(x,\lambda
) $ est int\'{e}grable ( mesurable suffisant car (3)) sur $X$.

\item Pour $\mu $-presque tout $x\in X$, la fonction $\lambda \longmapsto
f(x,\lambda )$\ est continue sur $\Lambda $

\item Il existe $g:X\rightarrow \mathbb{R}_{+}$ int\'{e}grable telle que $%
\forall \lambda \in \Lambda $ on ait $\left\vert f(x,\lambda )\right\vert
\leq g(x)$ pour $\mu $-presque tout $x\in X$
\end{itemize}

Alors la fonction $F:\Lambda \rightarrow \mathbb{C}$ d\'{e}finie par $%
F(\lambda )=\tint_{X}f(x,\lambda )dx$ est continue sur $\Lambda .$
\end{theorem}

\bigskip

\begin{remark}
Si l'on suppose seulement que $\lambda \longmapsto f(x,\lambda )$ est
continue en un point $\lambda _{0}\in \Lambda $, on obtient que $F$ est
continue en $\lambda _{0}.$
\end{remark}

\bigskip

\begin{theorem}[de d\'{e}rivabilit\'{e}]
On suppose

\begin{itemize}
\item $\forall \lambda \in \Lambda ,$ la fonction $x\longmapsto f(x,\lambda
) $ est int\'{e}grable sur $X$.

\item Pour $\mu $-presque tout $x\in X$, la fonction $\lambda \longmapsto
f(x,\lambda )$\ est d\'{e}rivable sur $\Lambda $

\item Il existe $g:X\rightarrow \mathbb{R}_{+}$ int\'{e}grable telle que
pour $\mu $-presque tout $x\in X,$ on ait $\left\vert \partial _{\lambda
}f(x,\lambda )\right\vert \leq g(x)$ $\forall \lambda \in \Lambda $
\end{itemize}

Alors l'application $F:\Lambda \rightarrow \mathbb{C}$ d\'{e}finie par $%
F(\lambda )=\tint_{X}f(x,\lambda )dx$ est d\'{e}rivable sur $\Lambda $ et 
\begin{equation*}
F^{^{\prime }}(\lambda )=\tint_{X}\partial _{\lambda }f(x,\lambda )dx
\end{equation*}
\end{theorem}

\bigskip

\begin{remark}
\begin{itemize}
\item $\partial _{\lambda }f(x,\lambda )$ est d\'{e}finie presque partout en 
$x$ et la ou elle ne l'est pas on lui met la valeur $0$.

\item Si $\Lambda =\left[ a,b\right] ,$ "derivable sur $\Lambda $"\ signifie
: derivable sur $\left] a,b\right[ $, derivable \`{a} droite en $a$ et d\'{e}%
rivable \`{a} gauch e en $b$.

\item Meme si on souhaite la d\'{e}rivabilit\'{e} de $F$ qu'en un point $%
\lambda _{0}\in \Lambda $, il faut quand meme supposer (3) pour $\forall
\lambda \in \Lambda .$
\end{itemize}
\end{remark}

\bigskip

\subsubsection{Epigraphe d'une fonction mesurable}

\bigskip 

Ici encore, on identifiera $\mathbb{R}^{d+1}$ avec $\mathbb{R}^{d}\times 
\mathbb{R}$ et si $x\in \mathbb{R}^{d}$, on notera $x_{1},...,x_{d}$ ses
coordonn\'{e}es.

\bigskip 

\begin{proposition}
Soit $f:\mathbb{R}^{d}\rightarrow \lbrack 0,+\infty ]$. On d\'{e}finit $%
E\subset \mathbb{R}^{d+1}$ par $E=\left\{ (x,t)\in \mathbb{R}^{d+1}\text{ }|%
\text{ }0\leq t\leq f(x)\right\} .$ $E$ est appel\'{e} l'\'{e}pigraphe de $f.
$ Alors $f$ est mesurable si et seulement si $E$ est mesurable et dans ce
cas 
\begin{equation*}
\lambda (E)=\tint_{\mathbb{R}^{d}}f(x)dx
\end{equation*}
\end{proposition}

\bigskip 

\subsection{Int\'{e}gration sur les espaces produits}

\bigskip

\subsubsection{Produit d'espaces mesurables}

Etant donn\'{e} deux espaces mesurables, on veut d\'{e}finir une tribu sur
le produit cart\'{e}sien de ces espaces. Soient $(X_{1},\mathcal{M}_{1})$ et 
$(X_{2},\mathcal{M}_{2})$ deux espaces mesurables. On munit le produit $%
X_{1}\times X_{2}$ de la tribu produit $\mathcal{M}_{1}\otimes \mathcal{M}%
_{2}$ d\'{e}finie par : $\mathcal{M}_{1}\otimes \mathcal{M}_{2}=\sigma
(A_{1}\times A_{2}$ $|$ $A_{1}\in \mathcal{M}_{1},A_{2}\in \mathcal{M}_{2}).$
$A_{1}\times \ A_{2}$, $A_{1}\in \mathcal{M}_{1},A_{2}\in \mathcal{M}_{2}$
est appel\'{e} un rectangle mesurable, et l'ensemble des rectangles
mesurables n'est pas une tribu en g\'{e}n\'{e}ral. Montrons quelques propri%
\'{e}t\'{e}s de la tribu produit :

\begin{itemize}
\item La tribu $\mathcal{M}_{1}\otimes \mathcal{M}_{2}$ est la plus petite
tribu sur $X_{1}\times X_{2}$ qui rende mesurable les deux projections
canoniques $\pi _{1}:X_{1}\times X_{2}\rightarrow X_{1}$ et $\pi
_{2}:X_{1}\times X_{2}\rightarrow X_{2}$.

\item Soit $(X,\mathcal{M})$ un autre espace mesurable, et soit $%
f=(f_{1},f_{2})$ une application de $X$ dans $X_{1}\times X_{2}$. Alors $%
f:(X,\mathcal{M})\rightarrow (X_{1}\times X_{2},\mathcal{M}_{1}\otimes 
\mathcal{M}_{2})$ est mesurable si et seulement si $f_{i}:(X,\mathcal{M}%
)\rightarrow (X_{i},\mathcal{M}_{i})$ est mesurable pour $i=1,2$.
\end{itemize}

\bigskip 

On \'{e}tend facilement la d\'{e}finition de la tribu produit pour un nombre
fini d'espaces mesurables $(X_{1},\mathcal{M}_{1}),...,(X_{n},\mathcal{M}%
_{n})$ en posant%
\begin{equation*}
\mathcal{M}_{1}\otimes \mathcal{M}_{2}\otimes \text{\textperiodcentered
\textperiodcentered \textperiodcentered }\otimes \mathcal{M}_{n}=\sigma
(A_{1}\times A_{2}\times \text{\textperiodcentered \textperiodcentered
\textperiodcentered }\times A_{n}\text{ }|\text{ }A_{i}\in \mathcal{M}%
_{i},\forall i\in \{1,2,...,n\})
\end{equation*}

et on a la propri\'{e}t\'{e} d'associativit\'{e}" suivante :%
\begin{equation*}
(\mathcal{M}_{1}\otimes \mathcal{M}_{2})\otimes \mathcal{M}_{3}=\mathcal{M}%
_{1}\otimes (\mathcal{M}_{2}\otimes \mathcal{M}_{3})=\mathcal{M}_{1}\otimes 
\mathcal{M}_{2}\otimes \mathcal{M}_{3}
\end{equation*}

\bigskip 

\begin{proposition}
Soient $X_{1}$ et $X_{2}$ deux espaces topologiques, munis de leur tribu bor%
\'{e}lienne. Alors 

\begin{itemize}
\item $\mathcal{B}(X_{1}\times X_{2})\supset \mathcal{B}(X_{1})\otimes 
\mathcal{B}(X_{2})$.

\item Si $X_{1}$ et $X_{2}$ sont des espaces m\'{e}triques s\'{e}parables,
alors on a :$B(X_{1}\times X_{2})\supset B(X_{1})\otimes B(X_{2})$.
\end{itemize}
\end{proposition}

\bigskip 

Introduisons les notations suivantes. Si $(X,\mathcal{M})$ et $(Y,\mathcal{N}%
)$ sont deux espaces mesurables quelconques et si $E\subset X\times Y$, on
note :%
\begin{eqnarray*}
E_{x} &=&\{y\in Y\text{ }|\text{ }(x,y)\in E\}\text{ pour }x\in X \\
E^{y} &=&\{x\in X\text{ }|\text{ }(x,y)\in E\}\text{ pour }y\in Y
\end{eqnarray*}%
Et si $f:X\times Y\rightarrow Z$ (espace quelconque), on d\'{e}finit 
\begin{eqnarray*}
f_{x} &:&Y\rightarrow Z\text{ par }fx(y)=f(x,y)\text{ pour }x\in X \\
f_{y} &:&X\rightarrow Z\text{ par }f_{y}(x)=f(x,y)\text{ pour }y\in Y
\end{eqnarray*}

\begin{proposition}
Soient $(X,\mathcal{M})$ et $(Y,\mathcal{N})$ deux espaces mesurables.

\begin{itemize}
\item $i)$ Si $E\in \mathcal{M}\otimes \mathcal{N}$, alors $E_{x}\in 
\mathcal{N}$, $\forall x\in X$ et $E^{y}\in \mathcal{M}$, $\forall y\in Y$ .

\item $ii)$ Si$f:X\times Y\rightarrow Z$ (espace mesurable quelconque) est
mesurable pour la tribu produit $\mathcal{M}\otimes \mathcal{N}$, alors%
\begin{eqnarray*}
\forall x &\in &X,\text{ }f_{x}:Y\rightarrow Z\text{ est mesurable} \\
\forall y &\in &Y,\text{ }f_{y}:X\rightarrow Z\text{ est mesurable}
\end{eqnarray*}
\end{itemize}
\end{proposition}

\subsubsection{Mesure Produit}

On veut maintenant construire une mesure sur l'espace produit X \times\ Y o%
\`{u} X et Y sont des espaces mesur\'{e}s.

\bigskip 

\begin{definition}
On dit qu'un espace mesur\'{e} $(X,\mathcal{M},\mu )$ est $\sigma $-fini
s'il existe une suite croissante de parties mesurables $E_{n}$ telle que $%
X=\bigcup_{n\in \mathbb{N}}$ $E_{n}$ et $\mu (E_{n})<\infty $ pour tout $%
n\in N$.
\end{definition}

\bigskip 

\begin{definition}
Soient $(X,\mathcal{M},\mu )$ et $(Y,\mathcal{N},\upsilon )$ deux espaces
mesur\'{e}s $\sigma $-finis. Alors :

\begin{itemize}
\item $i)$ Il existe une unique mesure $m$ sur $(X\times Y,\mathcal{M}%
\otimes \mathcal{N})$ telle que $m(A\times B)=\mu (A)\nu (B)$ $\forall A\in
M,\forall B\in N$

\item $ii)$ Pour tout $E\in \mathcal{M}\otimes \mathcal{N}$, on a : $%
m(E)=\int_{X}\nu (E_{x})d\mu _{x}=\int_{Y}\mu (E^{y})d\nu _{y}$
\end{itemize}
\end{definition}

\bigskip 

\begin{remark}

\begin{itemize}
\item Par le m\^{e}me proc\'{e}d\'{e}, on peut d\'{e}finir le produit de
nmesures $\sigma $-finies $\mu _{1},...,\mu _{n}$ en posant par exemple $\mu
_{1}\otimes $\textperiodcentered \textperiodcentered \textperiodcentered $%
\otimes \mu _{n}=\mu _{1}\otimes (\mu _{2}\otimes ($\textperiodcentered
\textperiodcentered \textperiodcentered $\otimes \mu _{n})).$ L'ordre des
parenth\`{e}ses n'a pas d'importance, car la mesure produit est enti\`{e}%
rement d\'{e}finie par sa valeur sur les pav\'{e}s mesurables : $(\mu
_{1}\otimes $\textperiodcentered \textperiodcentered \textperiodcentered $%
\otimes \mu _{n})(A_{1}\times $\textperiodcentered \textperiodcentered
\textperiodcentered $\times A_{n})=\mu _{1}(A_{1})$\textperiodcentered
\textperiodcentered \textperiodcentered $\mu _{n}(A_{n})$

\item Si $(X,\mathcal{M},\mu )=(Y,\mathcal{N},\upsilon )=(\mathbb{R},%
\mathcal{B}(\mathbb{R}),\lambda )$, alors $X\times Y=\mathbb{R}^{2},\mathcal{%
M}\otimes \mathcal{N}=\mathcal{B}(\mathbb{R}^{2})$ et $\mu \otimes \nu
=\lambda _{2}$ la mesure de Lebesgue sur $\mathbb{R}^{2}$. En effet, la derni%
\`{e}re \'{e}galit\'{e} provient du fait que les mesures prennent les m\^{e}%
mes valeurs sur les pav\'{e}s mesurables. De m\^{e}me, $\lambda _{d}=\lambda
_{1}\otimes $\textperiodcentered \textperiodcentered \textperiodcentered $%
\otimes \lambda _{1}$ ($d$ fois). Ainsi, on aurait pu se contenter de
montrer l'existence de la mesure de Lebesgue sur $\mathbb{R}$ et g\'{e}n\'{e}%
raliser sur $\mathbb{R}^{d}$ avec cette remarque.

\item L'hypoth\`{e}se de $\sigma $-finitude est n\'{e}cessaire. En effet,
soit $(X,\mathcal{M})=(Y,\mathcal{N})=(\mathbb{R},\mathcal{B}(\mathbb{R}%
)),\mu =\lambda $ la mesure de Lebesgue et $\upsilon $ la mesure de comptage
(non $\sigma $-finie). Soit $E=\left\{ (x,y)\in \mathbb{R}^{2}\text{ }|\text{
}x=y\right\} \in \mathcal{M}\otimes \mathcal{N}=\mathcal{B}(\mathbb{R}^{2}).$
On a $\upsilon (E_{x})=1$, $\forall x\in X$ et $\mu (E^{y})=0$, $\forall
y\in Y$. Or%
\begin{equation*}
\infty =\int_{X}\upsilon (E_{x})d\mu _{x}\neq \int_{Y}\mu (E^{y})d\upsilon
_{y}=0
\end{equation*}
\end{itemize}
\end{remark}

\subsubsection{Th\'{e}oremes de Fubini}

\bigskip 

\begin{theorem}[Fubini-Tonelli]
Soient $(X,\mathcal{M},\mu )$ et $(Y,\mathcal{N},\upsilon )$ deux espaces
mesur\'{e}s $\sigma $-finis, et soit $f:X\times Y\rightarrow \lbrack
0,+\infty ]$ une fonction $\mathcal{M}\otimes \mathcal{N}$-mesurable. Alors :

\begin{itemize}
\item les fonctions $%
\begin{array}{c}
(X,\mathcal{M)\rightarrow }\left[ 0,+\infty \right]  \\ 
x\mapsto \int_{Y}f(x,y)d\upsilon _{y}%
\end{array}%
$ et\ $%
\begin{array}{c}
(Y,\mathcal{N)\rightarrow }\left[ 0,+\infty \right]  \\ 
y\mapsto \int_{X}f(x,y)d\mu _{x}%
\end{array}%
$\ sont mesurables.

\item On a les \'{e}galit\'{e}s suivantes :%
\begin{equation*}
\int_{X\times Y}f(x,y)d(\mu \otimes \upsilon )=\int_{X}\left(
\int_{Y}f(x,y)d\upsilon _{y}\right) d\mu _{x}=\int_{Y}\left(
\int_{X}f(x,y)d\mu _{x}\upsilon _{y}\right) d\upsilon _{y}\in \left[
0,+\infty \right] 
\end{equation*}
\end{itemize}
\end{theorem}

\bigskip 

\begin{theorem}[Fubini-Lebesgue]
Soient $(X,\mathcal{M},\mu )$ et $(Y,\mathcal{N},\upsilon )$ deux espaces
mesur\'{e}s $\sigma $-finis, et soit $f:X\times Y\rightarrow \mathbb{R}$ ou $%
\mathbb{C}$ une fonction int\'{e}grable, i.e $f\in \mathcal{L}(X\times Y,%
\mathcal{M}\otimes \mathcal{N},\mu \otimes \upsilon )$. Alors :

\begin{itemize}
\item Pour presque tout $x\in X$ ( resp $y\in Y$\ ), la fonction $y\mapsto
f(x,y)$ ( resp $x\mapsto $\ ) est dans $\mathcal{L}^{1}(Y,\mathcal{N}%
,\upsilon )$ ( resp $\mathcal{L}^{1}(X,\mathcal{M},\mu )$ ).

\item La fonction $x\mapsto \int_{Y}f(x,y)d\upsilon _{y}$\ ( resp $y\mapsto $%
\ $\int_{X}f(x,y)d\mu _{x}$ )\ d\'{e}finie $\mu $ ( resp $\upsilon $\ )
-presque partout est dans $\mathcal{L}^{1}(X,\mathcal{M},\mu )$ ( resp $%
\mathcal{L}^{1}(Y,\mathcal{N},\upsilon )$ )

\item On a les \'{e}galit\'{e}s suivantes 
\begin{equation*}
\int_{X\times Y}f(x,y)d(\mu \otimes \upsilon )=\int_{X}\left(
\int_{Y}f(x,y)d\upsilon _{y}\right) d\mu _{x}=\int_{Y}\left(
\int_{X}f(x,y)d\mu _{x}\upsilon _{y}\right) d\upsilon _{y}
\end{equation*}
\end{itemize}
\end{theorem}

\bigskip 

\subsubsection{Compl\'{e}tion des mesures produits}

Soient $(X,\mathcal{M},\mu )$ et $(Y,\mathcal{N},\upsilon )$ deux espaces
mesur\'{e}s, $\sigma $-finis et complets ( leur mesure associ\'{e}e est compl%
\`{e}te ). En general, la mesure $\mu \otimes \upsilon $\ sur $X\times Y$\
n'est pas compl\`{e}te.

\bigskip 

\begin{proposition}
Soit $d\geq 2$ et $k,l$ des entiers tels que $d=k+l$. Alors la mesure de
Lebesgue $\lambda _{d}$ sur $\mathcal{L}(\mathbb{R}^{d})$ est la compl\'{e}t%
\'{e}e de la mesure produit $\lambda _{k}\otimes \lambda _{l}$\ sur $%
\mathcal{L}(\mathbb{R}^{d})\otimes \mathcal{L}(\mathbb{R}^{d})$.
\end{proposition}

\bigskip 

\begin{proposition}
Soient $(X,\mathcal{M},\mu )$ et $(Y,\mathcal{N},\upsilon )$ deux espaces
mesur\'{e}s, $\sigma $-finis et complets. Soit $f:X\times Y\rightarrow 
\mathbb{R}$ ou $\mathbb{C}$ une fonction mesurable pour la tribu produit
compl\'{e}t\'{e}e $(\mathcal{M}\otimes \mathcal{N})\ast $. Alors :

\begin{itemize}
\item $i)$ Le th\'{e}or\`{e}me de Fubini-Tonelli s'applique si $f\geq 0$, 
\`{a} la diff\'{e}rence pr\`{e}s que les fonction $f_{x}:y\mapsto f(x,y)$ et 
$f^{y}:x\mapsto f(x,y)$ sont simplement mesurables pour presque tous les $x$
ou $y$.

\item $ii)$ Le th\'{e}or\`{e}me de Fubini-Lebesgue s'applique tel qu'il a 
\'{e}t\'{e} \'{e}nonc\'{e}.
\end{itemize}
\end{proposition}

\subsection{Integration par partie et Changement de variable dans $\mathbb{R}%
^{d}$}

\bigskip

\bigskip

\bigskip

\bigskip

\bigskip

\section{Distribution}

\end{document}
