
\documentclass[3pt]{article}
%%%%%%%%%%%%%%%%%%%%%%%%%%%%%%%%%%%%%%%%%%%%%%%%%%%%%%%%%%%%%%%%%%%%%%%%%%%%%%%%%%%%%%%%%%%%%%%%%%%%%%%%%%%%%%%%%%%%%%%%%%%%
\usepackage{amssymb}
\usepackage{amsfonts}
\usepackage{amsmath}
\usepackage{fancyhdr}
\usepackage{geometry}
\usepackage{amscd}

\setcounter{MaxMatrixCols}{10}
%TCIDATA{OutputFilter=LATEX.DLL}
%TCIDATA{Version=4.10.0.2347}
%TCIDATA{Created=Wednesday, March 19, 2008 18:21:16}
%TCIDATA{LastRevised=Tuesday, November 12, 2019 17:31:02}
%TCIDATA{<META NAME="GraphicsSave" CONTENT="32">}
%TCIDATA{<META NAME="DocumentShell" CONTENT="Articles\SW\JEEP -  A General Purpose Vehicle">}
%TCIDATA{Language=American English}
%TCIDATA{CSTFile=LaTeX article (bright).cst}

\geometry{ hmargin=1.5cm, vmargin=1.5cm }
\newtheorem{theorem}{Theorem}
\newtheorem{acknowledgement}[theorem]{Acknowledgement}
\newtheorem{algorithm}[theorem]{Algorithm}
\newtheorem{axiom}[theorem]{Axiom}
\newtheorem{case}[theorem]{Case}
\newtheorem{claim}[theorem]{Claim}
\newtheorem{conclusion}[theorem]{Conclusion}
\newtheorem{condition}[theorem]{Condition}
\newtheorem{conjecture}[theorem]{Conjecture}
\newtheorem{corollary}[theorem]{Corollary}
\newtheorem{criterion}[theorem]{Criterion}
\newtheorem{definition}[theorem]{D\'{e}finition}
\newtheorem{example}[theorem]{Example}
\newtheorem{exercise}[theorem]{Exercise}
\newtheorem{lemma}[theorem]{Lemme}
\newtheorem{notation}[theorem]{Notation}
\newtheorem{problem}[theorem]{Problem}
\newtheorem{proposition}[theorem]{Proposition}
\newtheorem{remark}[theorem]{Remarque}
\newtheorem{solution}[theorem]{Solution}
\newtheorem{summary}[theorem]{Summary}
\newenvironment{Proof}[1][Proof]{\noindent\textbf{Proof} }{\ \rule{0.5em}{0.5em}}
\input{tcilatex}

\begin{document}

\title{Math}
\author{T.\ Monedero \\
%EndAName
Natixis Fixed Income Department:\\
quantitative analysis }
\maketitle

\begin{abstract}
\end{abstract}

\tableofcontents

\bigskip 

\bigskip 

\bigskip 

\bigskip 

\bigskip 

\bigskip 

\bigskip 

\bigskip 

\bigskip 

\bigskip 

\bigskip 

\bigskip 

\bigskip 

\bigskip 

\bigskip 

\bigskip 

\bigskip 

\bigskip 

\section{Mesure et Integration}

\bigskip

\subsection{Mesure}

\subsubsection{Espace Mesurable}

\bigskip 

\begin{definition}
Soit $X$ un ensemble. On appelle tribu ou $\sigma $- alg\`{e}bre sur $X$ une
famille $\mathcal{M}$ de partie de $X$ poss\'{e}dant les propri\'{e}t\'{e}s
suivantes :

\begin{itemize}
\item  $X\in \mathcal{M}$.

\item Si $A\in \mathcal{M}$, alors $A^{C}\in \mathcal{M}$ ( ou $A^{C}=X$ $%
\backslash $ $A$ est le compl\'{e}mentaire de $A$ dans $X$ ).

\item Si $A_{n}\in \mathcal{M}$, $\forall n\in \mathbb{N}$, alors $%
\tbigcup\limits_{n\in \mathbb{N}}A_{n}\in \mathcal{M}$.

Les \'{e}l\'{e}ments de $\mathcal{M}$ sont appel\'{e}s les parties
mesurables de $X$. On dit que ($X,\mathcal{M}$) est un espace mesurable.
\end{itemize}
\end{definition}

\bigskip 

$\mathcal{M=}\left\{ \varnothing ,X\right\} $ est la plus petite tribu de $X$
et $\mathcal{M=}\mathcal{P}(X)$ la plus grande. De plus, $\mathcal{M}$ est
stable par union ou intersection finie. En effet, si $A_{n}\in \mathcal{M}$, 
$\forall n\in \mathbb{N}$, alors $\tbigcap\limits_{n\in \mathbb{N}}A_{n}\in 
\mathcal{M}$ car $\left( \tbigcap\limits_{n\in \mathbb{N}}A_{n}\right)
^{C}=\tbigcup\limits_{n\in \mathbb{N}}A_{n}^{C}$. Enfin, si $A$ et $B$ sont
mesurables, alors la diff\'{e}rence non sym\'{e}trique $A$ $\backslash $ $%
B=A\tbigcap B^{C}\in \mathcal{M}$.

\bigskip 

\begin{lemma}
Soit $\left\{ \mathcal{M}_{i}\right\} _{i\in I}$ une famille quelconque de
tribus sur $X$. Alors $\mathcal{M=}\tbigcap_{i\in I}\mathcal{M}_{i}$ est
encore une tribu sur $X$.
\end{lemma}

\bigskip 

\begin{definition}
Soit $F$ une famille de parties de $X\ $et $\left\{ \mathcal{M}%
_{i}^{F}\right\} _{i\in I}$ la famille de tribus sur $X$ contenant $F$ ( i.e 
$\forall i\in I$, $F\subset \mathcal{M}_{i}^{F}$ ). On note 
\begin{equation*}
\sigma (F)=\tbigcap_{i\in I}\mathcal{M}_{i}^{F}
\end{equation*}%
la tribu engendr\'{e}e par $F$ sur $X$. C'est le plus petite tribu sur $X$
qui contient $F$.
\end{definition}

\bigskip 

\begin{definition}
Unt topologie sur $X\ $est une famille $\mathcal{T}$ de parties de $X$
telles que :

\begin{itemize}
\item $\varnothing \in \mathcal{T},$ $X\in \mathcal{T}$.

\item Si $O_{1},...,O_{n}\in \mathcal{T},$ alors $\tbigcap%
\limits_{i=1}^{n}O_{i}\in \mathcal{T}$.

\item Si $\left\{ O_{i}\right\} _{i\in I}$\ est une famille quelconque d'%
\'{e}l\'{e}ments de $\mathcal{T}$ alors $\tbigcup_{i\in I}O_{i}\in \mathcal{T%
}$.

Les \'{e}l\'{e}ments de $\mathcal{T}$ s'appelent les ouverts de $X$. On dit
que ( $X$, $\mathcal{T}$ ) est un espace topologique
\end{itemize}
\end{definition}

\bigskip 

\begin{definition}
Soit ( $X$, $\mathcal{T}$ ) un espace topologique. On appelle tribu de Borel
sur $X$ la tribu engendr\'{e}e par les ouverts de $X$ : $\mathcal{M}=\sigma (%
\mathcal{T})$ . La tribu $\mathcal{B(}\mathbb{R}\mathcal{)}$ est engendr\'{e}%
e par les intervalles $\left] a,+\infty \right[ $ pour $a\in \mathbb{R}.$
\end{definition}

\bigskip \bigskip 

\bigskip 

\bigskip 

\bigskip 

\subsubsection{Mesure Positive}

\bigskip 

\begin{definition}
Soit ($X,\mathcal{M}$) un espace mesurable. On appelle mesure positive sur $X
$ une application $\mu :\mathcal{M\rightarrow }\left[ 0,+\infty \right[ $
verifiant : 

\begin{itemize}
\item $\mu (\varnothing )=0$

\item Additivit\'{e} d\'{e}nonbrable : si $\left\{ A_{n}\right\} _{n\in 
\mathbb{N}}$ est une famille d\'{e}nombrable d'ensembles mesurables deux a
deux disjoints alors%
\begin{equation*}
\mu \left( \tbigcup_{n\in \mathbb{N}}A_{n}\right) =\tsum_{n\in \mathbb{N}%
}\mu \left( A_{n}\right) 
\end{equation*}%
On dit que ($X,\mathcal{M},\mu $) est un espace mesur\'{e}.
\end{itemize}
\end{definition}

\bigskip 

\begin{proposition}
Une mesure positive poss\`{e}de les propri\`{e}t\'{e}s suivantes :

\begin{itemize}
\item Si $A,B\in \mathcal{M}$ et $A\subset B$, alors $\mu (A)\leqslant \mu
\left( B\right) $\ ( Monotonie ).

\item Si $A_{n}\in \mathcal{M}$, $\forall n\in \mathbb{N}$ alors $\mu \left(
\tbigcup_{n\in \mathbb{N}}A_{n}\right) \leqslant \tsum_{n\in \mathbb{N}}\mu
\left( A_{n}\right) $ ( Sous additivit\'{e} ).

\item Si $A_{n}\in \mathcal{M}$, $\forall n\in \mathbb{N}$ et si $%
A_{n}\subset A_{n+1},\forall n\in \mathbb{N}$ alors $\mu \left(
\tbigcup_{n\in \mathbb{N}}A_{n}\right) =\underset{n\rightarrow +\infty }{%
\lim }\mu \left( A_{n}\right) .$

\item Si $A_{n}\in \mathcal{M}$, $\forall n\in \mathbb{N}$ et si $%
A_{n}\supset A_{n+1},\forall n\in \mathbb{N}$ avec $\mu \left( A_{0}\right)
<\infty $ alors $\mu \left( \tbigcap_{n\in \mathbb{N}}A_{n}\right) =\underset%
{n\rightarrow +\infty }{\lim }\mu \left( A_{n}\right) .$
\end{itemize}
\end{proposition}

\bigskip 

\begin{theorem}
Il existe une unique mesure positive sur $(\mathbb{R},\mathcal{B(\mathbb{R})}%
)$, not\'{e}e $\lambda $, telle que 
\begin{equation*}
\lambda (\left] a,b\right[ )=b-a,\forall a,b\in \mathbb{R}\times \mathbb{R}%
\text{ }\backslash \text{ }a<b
\end{equation*}%
$\lambda $ est appell\'{e}e mesure de Lebesgue sur $\mathbb{R}$. La mesure
de Lebesgue est diffuse : $\forall x\in \mathbb{R}$, $\lambda (\left\{
x\right\} )=0$. Par cons\'{e}quent,%
\begin{equation*}
\lambda (\left] a,b\right[ )=\lambda (\left[ a,b\right[ )=\lambda (\left] a,b%
\right] )=\lambda (\left[ a,b\right] )=b-a,\text{ }a\leq b
\end{equation*}
\end{theorem}

\subsubsection{Completion de Mesure}

\bigskip 

\subsubsection{Mesure de Lebesgue}

\subsection{Th\'{e}orie de l'int\'{e}gration}

\bigskip 

\subsubsection{Fonction Mesurable}

\bigskip

\subsubsection{Integration Fonction Mesurable Positive}

\bigskip 

\bigskip 

\section{Distribution}

\bigskip

\end{document}
