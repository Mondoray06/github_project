
\documentclass[3pt]{article}
%%%%%%%%%%%%%%%%%%%%%%%%%%%%%%%%%%%%%%%%%%%%%%%%%%%%%%%%%%%%%%%%%%%%%%%%%%%%%%%%%%%%%%%%%%%%%%%%%%%%%%%%%%%%%%%%%%%%%%%%%%%%
\usepackage{amssymb}
\usepackage{amsfonts}
\usepackage{amsmath}
\usepackage{fancyhdr}
\usepackage{geometry}
\usepackage{amscd}

\setcounter{MaxMatrixCols}{10}
%TCIDATA{OutputFilter=LATEX.DLL}
%TCIDATA{Version=4.10.0.2347}
%TCIDATA{Created=Wednesday, March 19, 2008 18:21:16}
%TCIDATA{LastRevised=Monday, May 27, 2019 17:14:34}
%TCIDATA{<META NAME="GraphicsSave" CONTENT="32">}
%TCIDATA{<META NAME="DocumentShell" CONTENT="Articles\SW\JEEP -  A General Purpose Vehicle">}
%TCIDATA{Language=American English}
%TCIDATA{CSTFile=LaTeX article (bright).cst}

\geometry{ hmargin=1.5cm, vmargin=1.5cm }
\newtheorem{theorem}{Theorem}
\newtheorem{acknowledgement}[theorem]{Acknowledgement}
\newtheorem{algorithm}[theorem]{Algorithm}
\newtheorem{axiom}[theorem]{Axiom}
\newtheorem{case}[theorem]{Case}
\newtheorem{claim}[theorem]{Claim}
\newtheorem{conclusion}[theorem]{Conclusion}
\newtheorem{condition}[theorem]{Condition}
\newtheorem{conjecture}[theorem]{Conjecture}
\newtheorem{corollary}[theorem]{Corollary}
\newtheorem{criterion}[theorem]{Criterion}
\newtheorem{definition}[theorem]{D\'{e}finition}
\newtheorem{example}[theorem]{Example}
\newtheorem{exercise}[theorem]{Exercise}
\newtheorem{lemma}[theorem]{Lemme}
\newtheorem{notation}[theorem]{Notation}
\newtheorem{problem}[theorem]{Problem}
\newtheorem{proposition}[theorem]{Proposition}
\newtheorem{remark}[theorem]{Remarque}
\newtheorem{solution}[theorem]{Solution}
\newtheorem{summary}[theorem]{Summary}
\newenvironment{Proof}[1][Proof]{\noindent\textbf{Proof} }{\ \rule{0.5em}{0.5em}}
\input{tcilatex}

\begin{document}

\title{FRTB Time Series}
\author{T.\ Monedero \\
%EndAName
Natixis Fixed Income Department:\\
quantitative analysis }
\maketitle

\begin{abstract}
\end{abstract}

\tableofcontents

.

\bigskip

\bigskip

\bigskip

\bigskip

\bigskip

\bigskip

\bigskip

\bigskip

\bigskip

\bigskip

\bigskip

\section{Introduction}

The aim of this document is to provide some technical elements that are
required for the reconstruction of implied volatility time series according
to today's market conventions.

\subsection{Notations}

$\ \ \ \ \ D(t,T)$ : Stochastic discount factor at time $t$ for the maturity 
$T$.

\bigskip

$P(t,T)$ : $T$-maturity zero coupon bond value at time $t\leq T$.

\bigskip

$L(T_{R},T_{S,}T_{S}+\tau )$ : Simply compounded Libor rate resetting at
time $T_{R}$ for the start date $T_{S}$ and the tenor $\tau $.

\bigskip

$S(T_{R},T_{S,}T_{E})$ : Swap rate resetting at time $T_{R}$ for the start
date $T_{S}$ and the end date $T_{E}$.

\subsection{Fixed Income Instruments}

\subsubsection{Forward Rate Agreement}

A Forword Rate Agreement $(FRA)$ is a contract in which a fixed rate payment
is exchanged against a payment based on the $T_{R}$ spot Libor for the
period $\left[ T_{S},T_{S}+\tau \right] $ with payments made in $T_{S}+\tau
. $ The value of a standard payer $FRA$ at time $0\leq t\leq T_{R}$ for a
fixed rate $K$ is given by%
\begin{eqnarray*}
FRA(t) &=&\mathbb{E}_{t}^{Q}\left[ D(t,T_{S}+\tau )(L(T_{R},T_{S},T_{S}+\tau
)-K)\right] \\
&=&P(t,T_{S}+\tau )\mathbb{E}_{t}^{Q^{T_{S}+\tau }}\left[ \left(
L(T_{R},T_{S},T_{S}+\tau )-K\right) \right]
\end{eqnarray*}%
Its fair value at time $t$ is obtained for 
\begin{equation*}
K^{\ast }=F(t,T_{S,}T_{S}+\tau )\overset{\Delta }{=}\mathbb{E}%
_{t}^{Q^{T_{S}+\tau }}\left[ \left( L(T_{R},T_{S},T_{S}+\tau )\right) \right]
\end{equation*}%
and then rewrites as%
\begin{equation*}
FRA(t)=P(t,T_{S}+\tau )\left( F(t,T_{S,}T_{S}+\tau )-K\right)
\end{equation*}

$F(t,T_{S,}T_{S}+\tau )$ is called a forward rate and is by construction
martingale under the forward measure $Q^{T_{S}+\tau }.$

\subsubsection{Interest Rate Swap}

A standard Interest Rate Swap $(IRS)$ resetting at time $T_{R}$ for the
start date $T_{S}$ and the end date $T_{E}$ is equivalent to a N-period $FRA$
with $T_{R}=T_{R}^{1}$, $T_{S}=T_{S}^{1}$, $T_{E}=T_{S}^{N}+\tau $ and \ $N=%
\frac{T_{E}-T_{S}}{\tau }.$ Its value at time $0\leq t\leq T_{R}$ for a
fixed rate $K$ is given by

\begin{eqnarray*}
IRS(t) &=&\mathbb{E}_{t}^{Q}\left[ \tsum_{i=1}^{N}\tau
_{i}D(t,T_{S}^{i}+\tau )(L(T_{R}^{i},T_{S}^{i},T_{S}^{i}+\tau )-K)\right] \\
&=&\tsum_{i=1}^{N}\tau _{i}P(t,T_{S}^{i}+\tau )\mathbb{E}_{t}^{Q^{T_{S}^{i}+%
\tau }}\left[ L(T_{R}^{i},T_{S}^{i},T_{S}^{i}+\tau )-K\right]
\end{eqnarray*}

Its fair value at time $t$ is obtained for 
\begin{equation*}
K^{\ast }=S(t,T_{S,}T_{E})\overset{\Delta }{=}\frac{\tsum_{i=1}^{N}\tau
_{i}P(t,T_{S}^{i}+\tau )F(t,T_{S,}^{i}T_{S}^{i}+\tau )}{\tsum_{i=1}^{N}\tau
_{i}P(t,T_{S}^{i}+\tau )}
\end{equation*}

and then rewrites as%
\begin{equation*}
IRS(t)=A(t)\left( S(t,T_{S,}T_{E})-K\right)
\end{equation*}

$S(t,T_{S,}T_{E})$ is called a forward swap rate and is martingale under the
annuity measure $Q^{A}$ defined by the annuity process $A(t)=\tsum_{i=1}^{N}%
\tau _{i}P(t,T_{S}^{i}+\tau ).$

\subsection{Interest Rate Options (1D)}

\subsubsection{Interest Rate Guarantee}

A payer $IRG$ option (or Caplet) is a type of option that gives the option
holder the opportunity to enter into a payer $FRA$. 
\begin{eqnarray*}
V_{Caplet}(t) &=&\mathbb{E}_{t}^{Q}\left[ D(t,T)\left( FRA(T)\right) _{+}%
\right] \\
&=&\mathbb{E}_{t}^{Q}\left[ D(t,T)P(T,T_{S}+\tau )\left(
F(T,T_{S,}T_{S}+\tau )-K\right) _{+}\right] \\
&=&P(t,T_{S}+\tau )\mathbb{E}_{t}^{Q^{T_{S}+\tau }}\left[ \left(
F(T,T_{S,}T_{S}+\tau )-K\right) _{+}\right]
\end{eqnarray*}

$.$

\subsubsection{European Swaptions}

\paragraph{Physical settlement}

A payer swaption with physical settlement is an option giving the right to
enter into a payer $IRS$ \bigskip 
\begin{eqnarray*}
V_{Swaption}^{Physical}(t) &=&\mathbb{E}_{t}^{Q}\left[ D(t,T)\left(
IRS(T)\right) _{+}\right] \\
&=&\mathbb{E}_{t}^{Q}\left[ D(t,T)A(T)\left( S(T,T_{S,}T_{E})-K\right) _{+}%
\right] \\
&=&A(t)\text{ }\mathbb{E}_{t}^{A}\left[ \left( S(T,T_{S,}T_{E})-K\right) _{+}%
\right]
\end{eqnarray*}

\paragraph{Cash settlement}

The value of a payer swaption with cash settlement is given by \bigskip 
\begin{eqnarray*}
V_{Swaption}^{Cash}(t) &=&\mathbb{E}_{t}^{Q}\left[ D(t,T_{S})A_{c}(T)\left(
S(T,T_{S,}T_{E})-K\right) _{+}\right] \\
&=&\mathbb{E}_{t}^{Q}\left[ \mathbb{E}_{T}^{Q}\left[ D(T,T_{S})\right]
D(t,T)A_{c}(T)\left( S(T,T_{S,}T_{E})-K\right) _{+}\right] \\
&=&\mathbb{E}_{t}^{Q}\left[ P(T,T_{s})D(t,T)A_{c}(T)\left(
S(T,T_{S,}T_{E})-K\right) _{+}\right] \\
&\mathbb{=}&A(t)\mathbb{E}_{t}^{A}\left[ \frac{P(T,T_{s})A_{c}(T)}{A(T)}%
\left( S(T,T_{S,}T_{E})-K\right) _{+}\right]
\end{eqnarray*}%
with $A_{c}(T)=\tsum_{i=1}^{N}\frac{\tau _{i}}{(1+S_{T})^{\tau _{i}}}\simeq 
\frac{1}{S_{T}}\left( 1-\frac{1}{(1+S_{T})^{T_{E}-T_{S}}}\right) $ the cash
annuity process and $S_{T}=S(T,T_{S,}T_{E})$ the swap forward rate.

\bigskip

\bigskip There are several ways to value this payoff. The standard market
formula is based on the hypothesis of the low variance of the process $H_{t}$
define by 
\begin{equation*}
H_{t}=\frac{P(t,T_{s})A_{c}(t)}{A(t)}\approx \frac{P(T,T_{s})A_{c}(T)}{A(T)}
\end{equation*}%
to finally establish\bigskip 
\begin{equation*}
V_{Cash}^{Swaption}(t)\approx P(t,T_{p})A_{c}(t)\mathbb{E}_{t}^{A}\left[
\left( S(T,T_{S,}T_{E})-K\right) _{+}\right]
\end{equation*}

A cash-settlement swaption can also be priced using more sophisticated
approximations and static replication for example.

\subsection{Interest Rate Options (2D)}

\subsubsection{Spread Options}

\bigskip The payoff of a spread option is given by%
\begin{equation*}
\left( S_{T}^{1}-S_{T}^{2}-K\right) _{+}
\end{equation*}

where $S_{T}^{1}$, $S_{T}^{2}$ are two swap rates of differents tenors
fixing at time $T$ and starting at time $T_{S}$.

\bigskip

The value of a single look spread option (forward prenium and payment in $%
T_{S}$) at time $t$ is therefore given by%
\begin{eqnarray*}
V_{SLSO}(t) &=&\frac{1}{P(t,T_{S})}\mathbb{E}_{t}^{Q}\left[ D(t,T_{S})\left(
S_{T}^{1}-S_{T}^{2}-K\right) _{+}\right]  \\
&=&\mathbb{E}_{t}^{T_{S}}\left[ \left( S_{T}^{1}-S_{T}^{2}-K\right) _{+}%
\right] 
\end{eqnarray*}

The value of a standard spread option (spot prenium and payment at the end
of each periods) at time $t$ is  given by%
\begin{eqnarray*}
V_{SO}(t) &=&\mathbb{E}_{t}^{Q}\left[ \tsum\limits_{i=1}^{N}\tau
_{i}D(t,T_{E}^{i})\left( S_{T_{i}}^{1}-S_{T_{i}}^{2}-K\right) _{+}\right]  \\
&=&\tsum\limits_{i=1}^{N}\tau _{i}P(t,T_{E}^{i})\mathbb{E}_{t}^{T_{E}^{i}}%
\left[ \left( S_{T_{i}}^{1}-S_{T_{i}}^{2}-K\right) _{+}\right]  \\
&\cong &\tsum\limits_{i=1}^{N}\tau _{i}P(t,T_{E}^{i})\mathbb{E}%
_{t}^{T_{S}^{i}}\left[ \left( S_{T_{i}}^{1}-S_{T_{i}}^{2}-K\right) _{+}%
\right]  \\
&=&\tsum\limits_{i=1}^{N}\tau _{i}P(t,T_{E}^{i})V_{SLSO}^{i}(t),\text{ \ \ }%
\tau _{i}=T_{E}^{i}-T_{S}^{i}
\end{eqnarray*}

\section{ATM implied volatility}

\bigskip

\subsection{Model conversion}

We note $Q$ an arbitrary probability measure and $W_{t}$ the brownian motion
under this probability. We assume the underlying asset $X_{t}$ being a
martingale under $Q$.

\subsubsection{Bachelier's model}

In the Bachelier's model, the dynamics of $X_{t}$ is given by

\begin{equation*}
dX_{t}=\sigma _{N}\text{ }dW_{t}
\end{equation*}

with $\sigma _{N}$ the normal volatility of $X_{t}$ and the price of a $T$
maturity call option at time $t$ is given by

\begin{eqnarray*}
\mathbb{E}_{t}^{Q}\left[ \left( X_{T}-K\right) _{+}\right]
&=&C_{t}^{N}(\sigma _{N\text{ }},X_{t},K,T) \\
&=&\sigma _{N\text{ }}\sqrt{T-t}\text{ }\left( \text{ }d\text{ }\Phi \left(
d\right) +\phi (d)\right)
\end{eqnarray*}%
with

\begin{equation*}
d=\frac{X_{t}-K}{\sigma _{N}\text{ }\sqrt{T-t}}
\end{equation*}

\subsubsection{Black's model}

In the Black's model, the dynamics of $X_{t}$ is assumed to be

\begin{equation*}
dX_{t}=\sigma _{LN}\text{ }X_{t}\text{ }dW_{t}
\end{equation*}

with $\sigma _{LN}$ the lognormal volatility of $X_{t}$ and the price of a $%
T $ maturity call option at time $t$ is given by

\begin{eqnarray*}
\mathbb{E}_{t}^{Q}\left[ \left( X_{T}-K\right) _{+}\right]
&=&C_{t}^{LN}(\sigma _{LN\text{ }},X_{t},K,T) \\
&=&X_{t}\Phi (d_{1})-K\Phi (d_{2})
\end{eqnarray*}%
with

\begin{eqnarray*}
d_{1} &=&\frac{1}{\sigma _{LN}\text{ }\sqrt{T-t}}\left( \ln \left( \frac{%
X_{t}}{K}\right) +\frac{\sigma _{LN}^{2}}{2}\left( T-t\right) \right) \\
d2 &=&d1-\sigma _{LN}\sqrt{T-t}
\end{eqnarray*}

\bigskip

\bigskip

\bigskip

For the same option price, Bachelier and Black ATM implied volatilities are
related by :

\begin{eqnarray*}
C_{t}^{N}(\sigma _{N\text{ }},X_{t},X_{t},T) &=&\sigma _{N\text{ }}\sqrt{%
\frac{T-t}{2\pi }} \\
C_{t}^{LN}(\sigma _{LN\text{ }},X_{t},X_{t},T) &=&X_{t}.\left( 2\Phi \left( 
\frac{\sqrt{T-t}}{2}\sigma _{LN}\right) -1\right)
\end{eqnarray*}%
which is equivalent to

\begin{equation*}
\sigma _{N}=X_{t}.\left( 2\Phi \left( \frac{\sqrt{T-t}}{2}\sigma
_{LN}\right) -1\right) .\sqrt{\frac{2\pi }{T-t}}
\end{equation*}

\bigskip

\bigskip

\subsection{Stripping conversion}

\bigskip

\subsubsection{Monocurve framework}

\bigskip In the monocurve framework, we assume that the Libor rate
represents an investment based on the risk free rate. For the $T_{R}$ spot
Libor rate, in absence of arbitrage this assumption leads to%
\begin{equation*}
L(T_{R},T_{S},T_{S}+\tau )\overset{AOA}{=}\frac{1}{\tau }\left( \frac{%
P(T_{R},T_{S})}{P(T_{R},T_{S}+\tau )}-1\right)
\end{equation*}%
resulting in that the Libor process is a $Q^{T_{S}+\tau }$ martingale. In
other words, we have 
\begin{equation*}
F(t,T_{S,}T_{S}+\tau )\overset{\Delta }{=}\mathbb{E}_{t}^{Q^{T_{S}+\tau }}%
\left[ L(T_{R},T_{S},T_{S}+\tau )\right] =L(t,T_{S},T_{S}+\tau )=\frac{1}{%
\tau }\left( \frac{P(t,T_{S})}{P(t,T_{S}+\tau )}-1\right)
\end{equation*}%
Libor forward rates (and swap forward rates by extension) are just a
combination of spot Libor curve zero coupon bonds.

\subsubsection{Bicurve framework}

The Libor being an interbank interest rate, it has an imbedded credit risk
which makes unclear whether it is a suitable proxies for a risk free rate.
We must therefore use two curves (a) the libor\ pseudo discount\ curve\ used
to project the libor based floating cash flows\ and a real discount curve
used to discount all cash flows

\begin{eqnarray*}
&&P^{d}(t,T)\overset{}{=}\mathbb{E}_{t}^{Q_{d}}\left[ D(t,T)\right] \\
&&F(t,T_{S,}T_{S}+\tau )\overset{\Delta }{=}\mathbb{E}_{t}^{Q_{d}^{T_{S}+%
\tau }}\left[ L(T_{R},T_{S},T_{S}+\tau )\right] \overset{\Delta }{=}\frac{1}{%
\tau }\left( \frac{P^{f}(t,T_{S})}{P^{f}(t,T_{S}+\tau )}-1\right)
\end{eqnarray*}

where $Q_{d}^{T_{S}+\tau }$ stands for the forward neutral probability
associated with the numeraire $P^{d}(t,T_{S}+\tau ).$ If we assume that
option preniums are market invariants, we can using bachelier's
representation summurize the impact of stripping on the ATM implied
volatility as follow

\bigskip

\begin{itemize}
\item Irg option
\end{itemize}

\begin{equation*}
\sigma _{N}^{B}=\frac{P(t,T_{S}+\tau )}{P^{d}(t,T_{S}+\tau )}\sigma _{N}^{M}
\end{equation*}

\begin{itemize}
\item Swaption physical settlement
\end{itemize}

\begin{equation*}
\sigma _{N}^{B}=\frac{A^{M}(t)}{A^{B}(t)}\sigma _{N}^{M}=\frac{%
\tsum\nolimits_{i=1}^{N}\tau _{i}P(t,T_{S}^{i}+\tau )}{\tsum%
\nolimits_{i=1}^{N}\tau _{i}P^{d}(t,T_{S}^{i}+\tau )}\sigma _{N}^{M}
\end{equation*}

\begin{itemize}
\item Swaption cash settlement
\end{itemize}

\begin{equation*}
\sigma _{N}^{B}=\frac{P(t,T_{S}).A_{c}^{M}(t)}{P^{d}(t,T_{S}).A_{c}^{B}(t)}%
\sigma _{N}^{M}
\end{equation*}%
with%
\begin{equation*}
\frac{A_{c}^{M}(t)}{A_{c}^{B}(t)}=\frac{S_{t}^{B}}{S_{t}^{M}}.\frac{%
((1+S_{t}^{M})^{\tau }-1).(1+S_{t}^{B})^{\tau }}{((1+S_{t}^{B})^{\tau
}-1).(1+S_{t}^{M})^{\tau }}\overset{}{\simeq }1+\tau \frac{%
(S_{t}^{B}-S_{t}^{M})}{(1+\tau S_{t}^{M})}\simeq 1
\end{equation*}%
we finally get

\begin{equation*}
\sigma _{N}^{B}=\frac{P(t,T_{S})}{P^{d}(t,Ts)}\sigma _{N}^{M}
\end{equation*}

where superscripts $B,M$ stand for monocurve and bicurve framework and where 
$P(t,T)$ is the zero coupon bond in the monocurve framework.

\bigskip

\section{Smiled implied volatility}

\subsection{SABR Model}

The SABR model is a stochastic volatility model producing an estimate of the
implied volatility curve used as an input in Black's model and is described
by the following equations%
\begin{eqnarray*}
dF_{t} &=&\alpha _{t}F_{t}^{\beta }dW_{t} \\
d\alpha _{t} &=&\eta \alpha _{t}dZ_{t} \\
\mathbb{E}\left[ dW_{t}dZ_{t}\right] &=&\rho dt
\end{eqnarray*}

with $F_{t\text{ }}$ and $\alpha _{t}$ the forward rate and the volatility
processes. The parameters are

\bigskip

\begin{itemize}
\item $F=F_{0}$ the initial forward

\item $\alpha =\alpha _{0}$ the initial volatility

\item $\eta $ the volatility of volatility

\item $\beta $ the exponent of the forward rate

\item $\rho $ the correlation between the brownian motions $W_{t}$ and $%
Z_{t} $
\end{itemize}

$\bigskip $

The implied volatility provided by the SABR model is given by

\bigskip

\begin{equation*}
\sigma _{SABR}(\alpha ,\rho ,\eta ,F,K,T)=\frac{z}{\chi (z)}.\frac{\alpha
.\left( 1+(T-t).\left[ \frac{\xi ^{2}}{6}.\frac{\alpha ^{2}}{(F.K)^{2\xi }}+%
\frac{\alpha \rho \beta \eta }{4(F.K)^{\xi }}+\frac{2-3\rho ^{2}}{24}\eta
^{2}\right] \right) }{(F.K)^{\xi }.\left[ 1+\frac{\xi ^{2}}{6}\ln ^{2}\left( 
\frac{F}{K}\right) +\frac{\xi ^{4}}{120}\ln ^{4}\left( \frac{F}{K}\right) %
\right] }
\end{equation*}%
with

\begin{eqnarray*}
\xi &=&\frac{1-\beta }{2} \\
z &=&\frac{\eta }{\alpha }.(F.K)^{\xi }.\ln \left( \frac{F}{K}\right) \\
\chi (z) &=&\ln \left( \frac{\sqrt{1-2\rho z+z^{2}}+z-\rho }{1-\rho }\right)
\end{eqnarray*}

\bigskip

In practice the choice of $\beta $ has little effect and is generally set to 
$\frac{1}{2}.$ $\rho $ and $\eta $ represent respectively the skew and the
convexity of the volatility smile

\bigskip

\bigskip

\bigskip

\subsection{Calibration}

For a given maturity $T$ and a set of $M$ prices, the estimation of $\alpha
,\rho ,$and $\eta $ can be performed by minimizing the errors between the
model and market prices%
\begin{equation*}
I_{MSE}^{T}=\left( \tsum\nolimits_{i=1}^{M}\omega _{i}\left[
P(F,K_{i,}T)_{Mkt}-C_{0}^{LN}(F_{0},k_{i,}T,\sigma _{SABR}(\alpha ,\rho
,\eta ,F,K_{i},T))\right] ^{2}\right)
\end{equation*}

\bigskip Other issues should be considered in the choice of a calibration
norm, including the term structure smoothness and time stationarity of
parameters%
\begin{eqnarray*}
I_{TS} &=&\left( \tsum\nolimits_{i=1}^{N-1}\omega _{TS,i}^{\rho }\left[ \rho
_{i+1}(t)-\rho _{i}(t)\right] ^{2}+\omega _{TS,i}^{\eta }\left[ \eta
_{i+1}(t)-\eta _{i}(t)\right] ^{2}\right) \\
I_{\Delta } &=&\left( \tsum\nolimits_{i=1}^{N}\omega _{\Delta ,i}^{\rho } 
\left[ \rho _{i}(t)-\rho _{i}(t-1)\right] ^{2}+\omega _{\Delta ,i}^{\eta }%
\left[ \rho _{i}(t)-\rho _{i}(t-1)\right] ^{2}\right)
\end{eqnarray*}

The problem can be cast as a minimisation of the following function

\begin{equation*}
\left( \alpha ^{\ast },\rho ^{\ast },\eta ^{\ast }\right) =\underset{\left(
\alpha ,\rho ,\eta \right) }{\min }\left[ \omega _{MSE}.I_{MSE}+\omega
_{TS}.I_{TS}+\omega _{\Delta }.I_{\Delta }\right]
\end{equation*}

with $I_{MSE}=\tsum\limits_{i=1}^{N}I_{MSE}^{T_{i}}$. The weights $\omega
_{MSE}$, $\omega _{TS}$ and $\omega _{\Delta }$ determine relative
importance for achieving smoothness and time stationarity of parameters over
a good fit to market prices. In practice, due to the different order of
magnitude of these error functions, defining these weights and resolving
this global problem with an suitable solution is quite difficult.

\bigskip

\bigskip

One can solve this problem by a three step calibration procedure :

\begin{itemize}
\item (a) : For each maturity\ $T_{i}$, proceed to the first calibration
step by minimizing the precision error function $I_{MSE}^{T_{i}}$ and give a
confidence interval $\left[ P_{-}^{k},P_{+}^{k}\right] $ for each price.

\item (b) : Compute the rescaling factors
\end{itemize}

\begin{eqnarray*}
\xi \Delta &=&\frac{\max (I_{MSE},I_{TS},I_{\Delta })}{I_{\Delta }} \\
\xi _{TS} &=&\frac{\max (I_{MSE},I_{TS},I_{\Delta })}{I_{TS}} \\
\xi _{MSE} &=&\frac{\max (I_{MSE},I_{TS},I_{\Delta })}{I_{MSE}}
\end{eqnarray*}%
and set 
\begin{equation*}
I^{\ast }=\omega _{\Delta }.\xi \Delta .I_{\Delta }+\omega _{TS}.\xi
_{TS}.I_{TS}+\omega _{MSE}.\xi _{MSE}.I_{MSE}
\end{equation*}

\bigskip

\begin{itemize}
\item (c) : With a convex combination of $\omega _{MSE}$, $\omega _{TS}$ and 
$\omega _{\Delta }$, minimize the new problem
\end{itemize}

\begin{equation*}
\left( \alpha ^{\ast },\rho ^{\ast },\eta ^{\ast }\right) =\underset{\left(
\alpha ,\rho ,\eta \right) }{\min }\left[ I^{\ast }+M(I^{\ast
})\tsum\limits_{k=1}^{M.N}\delta _{\overset{-}{X}}^{k}\right]
\end{equation*}

\bigskip

where $\delta _{\overset{-}{X}}^{k}=\left\{ 
\begin{array}{c}
1\text{ if }P^{k}\notin \left[ P_{-}^{k},P_{+}^{k}\right] \\ 
0,\text{otherwise}%
\end{array}%
\right. $and $M(I^{\ast })>>I^{\ast }$. The confidence intervals can be
choosen arbitrarily or according to market bid/ask spread.

\bigskip

\bigskip

\section{Smiled implied correlation}

The convexity adjusted constant maturity swap rate is classicaly defined by%
\begin{equation*}
CMS_{t}\overset{\Delta }{=}\mathbb{E}_{t}^{T_{S}}\left[ S_{T}\right]
\end{equation*}

Using the fact that $CMS_{T}=S_{T}$, the spread option value rewrites as%
\begin{equation*}
V_{SO}(t)=P(t,T_{S})\mathbb{E}_{t}^{T_{S}}\left[ \left(
CMS_{T}^{1}-CMS_{T}^{2}-K\right) _{+}\right]
\end{equation*}

The gaussian model provides a convenient common langage for quoting spread
options and gives an analytical expression of the implied spread correlation 
$\rho (t,T,K)$ between $CMS^{1}$ and $CMS^{2}$and is described by%
\begin{eqnarray*}
dCMS_{t}^{1} &=&\sigma _{N}^{1}\text{ }dW_{t}^{1} \\
dCMS_{t}^{2} &=&\sigma _{N}^{2}\text{ }dW_{t}^{2}
\end{eqnarray*}%
sqf $CMS_{T}^{1}-CMS_{T}^{2}\overset{d}{=}N(CMS_{t}^{1}-CMS_{t}^{2},)$

\section{Results}

In order to reconstruct the implied volatility time series we make the
following assumptions

\begin{itemize}
\item The prenium of atm option are market invariants and are stripping
independant

\item $\rho $ and $\eta $ are smile relative parameters and are also
stripping independant
\end{itemize}

\subsection{ATM Implied Volatility}

\paragraph{Input}

\bigskip 
\begin{tabular}{c|c|c|c|c|c|}
\cline{2-6}
& MC & BC & BLACK & NOR & SOURCE \\ \hline
\multicolumn{1}{|c|}{OSW\_EUR} & 02/01/2007 & 21/07/2011 & $\phi $ & All & 
Icap \\ \hline
\multicolumn{1}{|c|}{OSW\_USD} & 02/01/2007 &  & $\phi $ & All & Icap \\ 
\hline
\multicolumn{1}{|c|}{IRG\_EUR} & 02/01/2007 & 19/10/2012 & 02/01/2007 & 
25/11/2013 & Summit \\ \hline
\multicolumn{1}{|c|}{IRG\_USD} & 02/01/2007 & 19/10/2012 & 02/01/2007 & 
25/11/2013 & Summit \\ \hline
\end{tabular}

\paragraph{Process}

\bigskip

\bigskip 
\begin{tabular}{c|c|c|c|}
\cline{2-4}
& MC --\TEXTsymbol{>}BC & BLACK --\TEXTsymbol{>} NOR & OTHER \\ \hline
\multicolumn{1}{|c|}{OSW\_EUR} & 02/01/2007\_21/07/2011 & $\phi $ & Spline
int for missing 12Y expiry \\ \hline
\multicolumn{1}{|c|}{OSW\_USD} & 02/01/2007 & $\phi $ & Spline int for
missing 12Y expiry \\ \hline
\multicolumn{1}{|c|}{IRG\_EUR} & 02/01/2007\_18/10/2012 & 
02/01/2007\_22/11/2013 & $\phi $ \\ \hline
\multicolumn{1}{|c|}{IRG\_USD} & 02/01/2007\_18/10/2012 & 
02/01/2007\_22/11/2013 & $\phi $ \\ \hline
\end{tabular}

\bigskip \FRAME{ftbphFU}{5.1906in}{3.128in}{0pt}{\Qcb{OSW EUR 10Y10Y}}{}{%
capture.png}{\special{language "Scientific Word";type
"GRAPHIC";maintain-aspect-ratio TRUE;display "USEDEF";valid_file "F";width
5.1906in;height 3.128in;depth 0pt;original-width 5.1353in;original-height
3.0831in;cropleft "0";croptop "1";cropright "1";cropbottom "0";filename
'Capture.PNG';file-properties "XNPEU";}}

\subsection{Smiled Implied Volatility}

\paragraph{Input}

\qquad \bigskip 
\begin{tabular}{|c|c|c|c|c|}
\hline
FWD\_ANNUITY & MC & BC & MDM & TOTEM \\ \hline
OSW\_EUR & 01/2007 &  & All & $\phi $ \\ \hline
OSW\_USD & 01/2007 &  & All & $\phi $ \\ \hline
IRG\_EUR & 01/2007 & 10/2012 & $\phi $ & All \\ \hline
IRG\_USD & 01/2007 & 10/2012 & $\phi $ & All \\ \hline
\end{tabular}

\paragraph{Process}

\ 
\begin{tabular}{|c|c|c|c|c|}
\hline
SABR\_PARAM & POND & PRICE & CALIB & OUTLIERS \\ \hline
OSW\_EUR & Vega\_r & Totem & $I_{MSE}$ (LM) & $I_{MSE}$ + $I_{TS}$+ $%
I_{\Delta }$ (GRG) \\ \hline
OSW\_USD & Vega\_r & Totem & $I_{MSE}$ (LM) & $I_{MSE}$ + $I_{TS}$+ $%
I_{\Delta }$ (GRG) \\ \hline
IRG\_EUR & Flat & Totem & $I_{MSE}$ (LM) & $I_{MSE}$ + $I_{TS}$+ $I_{\Delta
} $ (GRG) \\ \hline
IRG\_USD & Flat & Totem & $I_{MSE}$ (LM) & $I_{MSE}$ + $I_{TS}$+ $I_{\Delta
} $ (GRG) \\ \hline
\end{tabular}

\bigskip

\end{document}
