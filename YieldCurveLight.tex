\documentclass[10pt,a4paper]{report}
\usepackage[utf8]{inputenc}
\usepackage[final]{pdfpages}
\usepackage[T1]{fontenc}
\usepackage[english]{babel}
\usepackage{bbold}
\usepackage{graphicx}
\usepackage{array}
\usepackage{caption}
\usepackage{multirow}
\usepackage{verbatim}
\usepackage{multirow}
\usepackage{amsmath}
\newcommand\verbfile[1]{%
	\begingroup
		\let\do\@makeother\dospecials
		\obeyspaces\obeylines\ttfamily
		\input#1\relax
	\endgroup
}
\makeatother

\begin{document}

\title{Yield Curve Econometric}
\author{T.\ Monedero \\
%EndAName
Quantitative Research\\
Fixed Income, Natixis\\
}
\maketitle

\begin{abstract}
The aim of this document is to provide some technical elements required for
the Fundamental Review of the Trading Book and more specifically for the
Internal Models Approach of the Yield Curve Econometric in the Historical
Simulation framework.
\end{abstract}


\tableofcontents\newpage
\renewcommand\thesection {\Roman{section}}
\renewcommand{\thesubsubsection}{\arabic{subsubsection} }
\setlength{\footnotesep}{2em} 
\renewcommand{\footnoterule}{\hspace*{0em}\dotfill\hspace*{0em}} 



\section{Introduction}

Defined by the Basel Committee of Banking Supervision, the Fundamental
Review of the Trading Book (FRTB) aims to improve the Basel II.5 regulation
rules and to build a new market risk framework as a response to the
financial crisis. These new standards address a number of both qualitative
and quantitative issues such as\ capital arbitrage between booking and
trading books as well as under-capitalization of the trading book.

\bigskip 

\subsection{Standardised Approach}

The Standardised Approach (SA) refers to a set of general risk measurement
techniques proposed in the FRTB, giving a market risk overview of all
bankink institutions. Needed to be calculated and reported to the relevant
supervisor on a monthly basis, this method provide minimum capital
requirements as the sum of three metrics :

\bigskip

\begin{itemize}
\item the Sensitivities-Based Method (SBM)

\item the Default Risk Capital\ (DRC)

\item the Residual Risk Add-On\ (RRAO)
\end{itemize}

\bigskip

Through risk agregation rules, the SBM use the sensitivities of financial
instruments to a predefined risk factor list for each risk classes (Interest
Rate, Credit, Equity, Commodity and Foreign Exchange) to calculate the
delta, vega and curvature risk capital requirements. The DRC is intented to
capture jump-to-default risk that may not be captured by credit spread
shocks under the SBM. To address possible limitations in the SA, the RRAO is
introduced to ensure sufficient coverage of market risks.

\bigskip 

\subsection{Internal Models Approach}

In order to reduce capital requirements, banks can also use the Internal
Models Approach (IMA) - at trading desk level - for a more accurate measure
of their own market risks. This process, however, entails an additional
calculation cost and a more complex methodology which is subject to the
supervisory authority agreement. As a consequence, the suitability of an internal risk management
model is assessed through the two following steps :

\bigskip 

\begin{itemize}
\item A P\&L Attribution test to determine the appropriateness of the choosen risk factors in relation to the material drivers of
Actual P\&L

\item A Backtesting test to determine how well trading desk  risks  are captured  by the risk factors modelling 
\end{itemize}

\bigskip

At bank's level, a third test is apply in order to split eligible
trading desks risk factors into the modellable risk factors (MRF) set and
the non modelable risk factors (NMRF) set. As in the SA, total capital charge for market risk under the IMA is given by the sum of the following metrics : \\

\begin{itemize}
\item the capital requirement for MRF
\item the capital requirement for NMRF
\item the Default Risk Capital for internal model
\item the standardised capital charge for ineligible trading desks
\end{itemize}

\bigskip

In order to compare both approachs,  banks should perform quantitative studies to measure the impacts on their trading business.

\section{Yield Curve Econometric}

\bigskip distinguer ce qui est observable (risk factors) de ce que l'on
souhaite modeliser par des consideration techniques et functionelle fonction
appliquer sur risk factor. contamination d'observabilit\'{e} 
sur determination du systeme ???

Hypoth\`{e}se du zero coupon traitable et observabilit\'{e} que de quelques
instruments  systeme sous determin\'{e}

Projection des instruments de taux par des raisonement d'arbitrage et des
definition propre a chaque banque et algo Diffusion de la courbe de taux

 Le vrai risque facteur : courbe des instruments de
taux.(justification par donn\'{e}e re\'{e}llement observables).

Interest rate modelling purpose is to describe the random dynamic of a set
of zero coupon bond curves through time starting from an initial condition.
Yield curve represents the global (majeur le plus important) risk for all
interest rate instruments including derivatives by extension.

\subsection{Par-Point}

In this approach, all pillars of the yield curve are considered independents
and the econometrics are directly applied to their facial values. Now, the
question is how to compare theses values at different dates. In fact, (
selon la date d'observation et la description des instruments (business days
vs calendar days), on peut avoir d'es instrument differents)

\bigskip

The first way is to extract zero coupon curves from past dates and use them
to value the current yield curve. (ajouter graph)$\times D$

\bigskip

\bigskip

\bigskip

The second way is to consider that rates with sliding maturities such as
money market and swap rates, can be compared regardless the date of
observation. For other rates induced by rolling securities such as Euribor
futures whose maturities are fixed dates, an immediate comparison is
impossible. An alternative can be to compute the current value of theses
rates from the past yield curves using some interpolation rules.

\bigskip

\bigskip

\bigskip

\bigskip

\bigskip

\bigskip

\bigskip

\bigskip

\bigskip

\bigskip

\bigskip

\bigskip

\bigskip

\bigskip

\bigskip

\bigskip

\bigskip

\bigskip

\bigskip

\bigskip

\bigskip

\bigskip

\bigskip

\bigskip

\bigskip

\bigskip

\bigskip

\bigskip

\bigskip

\bigskip

\bigskip

\section{Historical Simulation Framework}

Historical simulation is a method which assumes that probability law of a
set of variable rely on their past evolution, allowing to capture stylized
facts of financial time series such as fat tails or volatility clustering
without modelling assumptions. For a given time horizion $\alpha $ and a
risk factor $R$ defined on a domain $D_{R}$ and\ whose spot level at time $t$
- taken to represents one business day- is noted $R_{t}$, this translates
into

\begin{equation*}
R_{t+1}^{s,s+h}=f(R_{t},R_{s+h},R_{s}),\text{ \ \ }t-\alpha \leq s<t-h
\end{equation*}

where the shock function $f$ specifies how to measure past return between $s$
and $s+h$ and how to apply it between $t$ and $t+1$.


\subsection{Shock function caracterisation}
Beyond statistical
considerations, the following statements gives necessaries criterias to
define such functions on a set $D$ :

\bigskip

\begin{itemize}
\item The image of $f$ is $D.$

\item A total order exist on $D.$

\item The order relation between $R_{s+h}$ and $R_{s}$is the same as between 
$R_{t+1}^{s,s+h}$ and $R_{t}.$
\end{itemize}

\bigskip 

Through this caracterisation, it is possible to define major shock functions
on their domain of application.

\subsubsection{Absolute shock}

Absolute return $\delta _{A}^{s,s+h}$ between $s$ and $s+h$ is defined on $%
\mathbb{R}$ as%
\begin{equation*}
\delta _{A}^{s,s+h}=R_{s+h}-R_{s}
\end{equation*}%
The shocked value using absolute shift is obtened by 
\begin{equation*}
R_{t+1}^{s,s+h}=R_{t}+\delta _{A}
\end{equation*}

\subsubsection{Relative shock}

The reletive return $\delta _{R}^{s,s+h}$ between $s$ and $s+h$ is defined
on $\mathbb{R}_{+}^{\ast }$ by%
\begin{equation*}
\delta _{R}^{s,s+h}=R_{t}.\frac{\left( R_{s+h}-R_{s}\right) }{R_{s}}
\end{equation*}%
The shocked value using relative shift is obtened by 
\begin{equation*}
R_{t+1}^{s,s+h}=R_{t}+\delta _{R}^{s,s+h}
\end{equation*}

\subsubsection{Mixed shock}

The most intuitive way to define a mixed -or absolute-relative- return is to
use a convexe combination between absolute and relative shocks 
\begin{equation*}
\forall \text{ }\lambda \in \left[ 0,1\right] ,\text{ \ \ }\delta
_{AR}=\lambda .\delta _{A}+(1-\lambda ).\delta _{R}
\end{equation*}

It is\ not clear, however, to find a consistent domain with the above
caracterisation (first statement break on $\mathbb{R}_{+}^{\ast }$ and third
statement break on $\mathbb{R}$). Using the function $\phi _{p}$ defined $%
\forall $ $p$ $\in \left[ 0,1\right] $ by%
\begin{eqnarray*}
\phi _{p} &:&\mathbb{R}_{+}^{\ast }\rightarrow \mathbb{R} \\
x &\rightarrow &y=p.x+(1-p).\ln (x)
\end{eqnarray*}%
and its inverse 
\begin{eqnarray*}
\phi _{p}^{-1} &:&\mathbb{R}\rightarrow \mathbb{R}_{+}^{\ast } \\
y &\rightarrow &x=\left\{ 
\begin{array}{c}
y,\text{ \ \ \ \ \ \ \ \ \ \ \ \ \ \ \ \ \ \ \ \ \ \ \ \ \ \ \ \ \ \ \ \ }p=1
\\ 
\\ 
\exp (y),\text{ \ \ \ \ \ \ \ \ \ \ \ \ \ \ \ \ \ \ \ \ \ \ \ \ \ }p=0 \\ 
\\ 
\left( \frac{1-p}{p}\right) .W\left( \frac{p}{1-p}.\exp \left( \frac{y}{1-p}%
\right) \right) ,\text{ }p\in \left] 0,1\right[%
\end{array}%
\right.
\end{eqnarray*}

where $W$ is the Lambert function, it is possible to construct a mixed
return function on $\mathbb{R}_{+}^{\ast }$ through the following algorithm :

\bigskip

\begin{itemize}
\item Compute $R_{t}^{p},R_{s+h}^{p},$ and $R_{s}^{p}$ with 
\begin{equation*}
R^{p}=\phi _{p}(R)
\end{equation*}

\item Compute $R_{t+1}^{p}$ using absolute return 
\begin{equation*}
R_{t+1}^{p}=R_{t}^{p}+R_{s+h}^{p}-R_{s}^{p}
\end{equation*}

\item Retrieve $R_{t+1}$ and $\delta _{AR}$ by 
\begin{eqnarray*}
R_{t+1} &=&\phi _{p}^{-1}(R_{t+1}^{p}) \\
\delta _{AR} &=&R_{t+1}-R_{t}
\end{eqnarray*}
\end{itemize}

\subsection{Bounded risk factors scaling}

Another issue that needs to be tackled is related to bounded risk factors.
To meet shock function domain requirements, rescaling function of the form 
\begin{equation*}
\zeta :[m,M]\rightarrow \mathbb{K}
\end{equation*}

with $\mathbb{K}$ = $\mathbb{R}$ or $\mathbb{R}_{+}^{\ast }$ can be used. In
order to easily retieve risk factor after shock application and to avoid
brute force truncature, rescaling function should be at least bijective with
analitical inverse function. Noting $f_{\mathbb{K}}$ a suitable shock
function on $\mathbb{K}$ and $R_{t}^{\zeta }=\zeta (R_{t)}$ the rescaling
risk factor at time $t,$ the shocked risk factor is obtained by

\begin{equation*}
R_{t+1}=\zeta ^{-1}\circ f_{\mathbb{K}}(R_{t}^{\zeta },R_{s+h}^{\zeta
},R_{s}^{\zeta })
\end{equation*}%
To limit composition function impact on the econometric, other features can
be take into consideration such that

\begin{eqnarray*}
\zeta _{\alpha _{t},\beta _{t},\gamma _{t}}(R_{t}) &=&R_{t} \\
\zeta _{\alpha _{t},\beta _{t},\gamma _{t}}^{\prime }(R_{t}) &=&1 \\
\zeta _{\alpha _{t},\beta _{t},\gamma _{t}}^{\prime \prime }(R_{t}) &=&0
\end{eqnarray*}

\end{document}
