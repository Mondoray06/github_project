\documentclass[10pt,a4paper]{report}
\usepackage[utf8]{inputenc}
\usepackage[final]{pdfpages}
\usepackage[T1]{fontenc}
\usepackage[english]{babel}
\usepackage{bbold}
\usepackage{graphicx}
\usepackage{array}
\usepackage{caption}
\usepackage{multirow}
\usepackage{verbatim}
\usepackage{multirow}
\usepackage{amsmath}
\newcommand\verbfile[1]{%
	\begingroup
		\let\do\@makeother\dospecials
		\obeyspaces\obeylines\ttfamily
		\input#1\relax
	\endgroup
}
\makeatother

\begin{document}

\title{Yield Curve Econometric}
\author{T.\ Monedero \\
%EndAName
Quantitative Research\\
Fixed Income, Natixis\\
}
\maketitle

\begin{abstract}
The aim of this document is to provide some technical elements required for
the Fundamental Review of the Trading Book and more specifically for the
Internal Models Approach of the Yield Curve Econometric in the Historical
Simulation framework.
\end{abstract}


\tableofcontents\newpage
\renewcommand\thesection {\Roman{section}}
\renewcommand{\thesubsubsection}{\arabic{subsubsection} }
\setlength{\footnotesep}{2em} 
\renewcommand{\footnoterule}{\hspace*{0em}\dotfill\hspace*{0em}} 



\section{Introduction}

Defined by the Basel Committee of Banking Supervision, the Fundamental
Review of the Trading Book (FRTB) aims to improve the Basel II.5 regulation
rules and to build a new market risk framework as a response to the
financial crisis. These new standards address a number of both qualitative
and quantitative issues such as\ capital arbitrage between booking and
trading books as well as under-capitalization of the trading book.

\bigskip 

\subsection{Standardised Approach}

The Standardised Approach (SA) refers to a set of general risk measurement
techniques proposed in the FRTB, giving a market risk overview of all
bankink institutions. Needed to be calculated and reported to the relevant
supervisor on a monthly basis, this method provide the minimum capital
requirements as the sum of three metrics :

\bigskip

\begin{itemize}
\item the Sensitivities-Based Method (SBM)

\item the Default Risk Capital\ (DRC)

\item the Residual Risk Add-On\ (RRAO)
\end{itemize}

\bigskip

Through risk agregation rules, the SBM uses the sensitivities of financial
instruments to a predefined risk factor list for each asset class (Interest
Rates, Credit, Equities, Commodities and Foreign Exchange) to calculate the
delta, vega and curvature risk capital requirements. The DRC is intented to
capture jump-to-default risk that may not be captured by credit spread
shocks under the SBM. To address possible limitations in the SA, the RRAO is
introduced to ensure sufficient coverage of market risks.

\bigskip 

\subsection{Internal Models Approach}

In order to reduce capital requirements, banks can also use the Internal
Models Approach (IMA) - at trading desk level - for a more accurate measure
of their own market risks. This process, however, entails an additional
calculation cost and a more complex methodology which is subject to the
supervisory authority agreement. As a consequence, the suitability of an internal risk management
model is assessed through the two following steps :

\bigskip 

\begin{itemize}
\item A P\&L Attribution test to determine the suitability of the chosen risk factors in relation to the material drivers of
Actual P\&L

\item A Backtesting to determine how well trading desk  risks  are captured  by the risk factors modelling 
\end{itemize}

\bigskip

At bank's level, a third test is applied in order to split eligible
trading desks risk factors into the modellable risk factors (MRF) set and
the non modelable risk factors (NMRF) set. As in the SA, total capital charge for market risk under the IMA is given by the sum of the following metrics : \\

\begin{itemize}
\item the capital requirement for MRF
\item the capital requirement for NMRF
\item the Default Risk Capital for internal model
\item the standardised capital charge for ineligible trading desks
\end{itemize}

\bigskip

In order to compare both approaches,  banks should perform quantitative studies to measure the impacts on their trading business.

\newpage
\section{Yield Curve Econometric}

Generally speaking, knowledge of the current discount curve is a prerequisite for the valuation of interest rate derivatives. In practice, only a few discount bonds are directly observable in the market. On the basis of theoretical and technical considerations, discount bonds prices can, however, be derived from the prices of liquid securities such as money market  through a bootstrapping algorithm. The benchmark set of instruments used in this process is called Yield Curve and is the main risk factor of the interest rate asset class.


\subsection{Par-Point}

In this approach, all pillars of the yield curve are considered to be independent of each
and the econometrics are directly applied to their facial values. Now, the
question is how to compare theses values at different dates. In fact, (
selon la date d'observation et la description des instruments (business days
vs calendar days), on peut avoir d'es instrument differents)

\bigskip

The first way is to extract zero coupon curves from past dates and use them
to value the current yield curve. (ajouter graph)$\times D$

\bigskip

\bigskip

\bigskip

The second way is to consider that rates with sliding maturities such as
money market and swap rates, can be compared regardless the date of
observation. For other rates induced by rolling securities such as Euribor
futures whose maturities are fixed dates, an immediate comparison is
impossible. An alternative can be to compute the current value of theses
rates from the past yield curves using some interpolation rules.

\bigskip

\bigskip

\bigskip

\bigskip

\bigskip

\bigskip

\bigskip

\bigskip

\bigskip

\bigskip

\bigskip

\bigskip

\bigskip

\bigskip

\bigskip

\bigskip

\bigskip

\bigskip

\bigskip

\bigskip

\bigskip

\bigskip

\bigskip

\bigskip

\bigskip

\bigskip

\bigskip

\bigskip

\bigskip

\bigskip

\bigskip

\section{Historical Simulation Framework}

Historical simulation is a method which assumes that the distribution of a
set of variables relies on their past evolution, allowing to capture stylized
facts of financial time series such as fat tails or volatility clustering
without modelling assumptions. For a given time horizion $\alpha $ and a
risk factor $R$ defined on a domain $D_{R}$ and\ whose spot level at time $t$
- taken to represent one business day- is noted $R_{t}$, this translates
into

\begin{equation*}
R_{t+1}^{s,s+h}=f(R_{t},R_{s+h},R_{s}),\text{ \ \ }t-\alpha \leq s<t-h
\end{equation*}

where the shock function $f$ specifies how to measure past return between $s$
and $s+h$ and how to apply it between $t$ and $t+1$.


\subsection{Shock function characterisation}
Beyond statistical
considerations, the following statements give necessary criterias to
define such functions on a set $D$ :

\bigskip

\begin{itemize}
\item A total order exists on $D.$
\item The image of $D^{3}$ under $f$ is $D.$
\item $\forall $ $(R_{t},R_{s+h},R_{s})\in D^{3}$, the order relationship
between $R_{s+h}$ and $R_{s}$is the same as between $R_{t+1}^{s,s+h}$ and $%
R_{t}.$
\end{itemize}

\bigskip 

Through this caracterisation, it is possible to define major classes of shock functions.

\subsubsection{Absolute shock}

Absolute return $\delta _{A}^{s,s+h}$ between $s$ and $s+h$ is defined as%
\begin{equation*}
\delta _{A}^{s,s+h}=R_{s+h}-R_{s}
\end{equation*}%
Absolute shock function is relevant for risk factors whose domain is $%
\mathbb{R}$ and the shocked value is given by 
\begin{equation*}
R_{t+1}^{s,s+h}=R_{t}+\delta _{A}
\end{equation*}

\subsubsection{Relative shock}

The reletive return $\delta _{R}^{s,s+h}$ between $s$ and $s+h$ is defined as%
\begin{equation*}
\delta _{R}^{s,s+h}=R_{t}.\frac{\left( R_{s+h}-R_{s}\right) }{R_{s}}
\end{equation*}%
Absolute shock function is relevant for risk factors whose domain is $%
\mathbb{R}_{+}^{\ast }$ and the shocked value is given by 
\begin{equation*}
R_{t+1}^{s,s+h}=R_{t}+\delta _{R}^{s,s+h}
\end{equation*}

\subsubsection{Mixed shock}

The most intuitive way to define a mixed return is to use a convex
combination between the absolute and relative shocks 
\begin{equation*}
\forall \text{ }\lambda \in \left[ 0,1\right] ,\text{ \ \ }\delta
_{AR}=\lambda .\delta _{A}+(1-\lambda ).\delta _{R}
\end{equation*}

It is unclear, however, how to find a consistent domain with the above
characterisation - $\mathbb{R}_{+}^{\ast }$ and $\mathbb{R}$ sets are not
suitable with regards to the first and third statements -.\\ 

\bigskip


Using the function $\phi _{p}$ defined $\forall $ $p$ $\in \left[ 0,1\right[ 
$ by%
\begin{eqnarray*}
\phi _{p} &:&\mathbb{R}_{+}^{\ast }\rightarrow \mathbb{R} \\
x &\rightarrow &y=p.x+(1-p).\ln (x)
\end{eqnarray*}%
and its inverse 
\begin{eqnarray*}
\phi _{p}^{-1} &:&\mathbb{R}\rightarrow \mathbb{R}_{+}^{\ast } \\
y &\rightarrow &x=\left\{ 
\begin{array}{c}
\exp (y),\text{ \ \ \ \ \ \ \ \ \ \ \ \ \ \ \ \ \ \ \ \ \ \ \ \ \ }p=0 \\ 
\\ 
\left( \frac{1-p}{p}\right) .W\left( \frac{p}{1-p}.\exp \left( \frac{y}{1-p}%
\right) \right) ,\text{ }p\in \left] 0,1\right[ 
\end{array}%
\right. 
\end{eqnarray*}

where $W$ is the Lambert function, it is possible to construct a mixed shock
function on $\mathbb{R}_{+}^{\ast }$ through the following steps :

\bigskip

\begin{itemize}
\item Compute $R_{t}^{p},R_{s+h}^{p},$ and $R_{s}^{p}$ with 
\begin{equation*}
R^{p}=\phi _{p}(R)
\end{equation*}

\item Compute $R_{t+1}^{p}$ using absolute return 
\begin{equation*}
R_{t+1}^{p}=R_{t}^{p}+R_{s+h}^{p}-R_{s}^{p}
\end{equation*}
\end{itemize}

\bigskip 

Mixed return is then given by $ \delta _{AR} =R_{t+1}-R_{t} $ with $ R_{t+1} =\phi _{p}^{-1}(R_{t+1}^{p}) $

\newpage

\subsection{Scaling function characterisation}

Before computing a shock, bounded risk factors must be adapted to the domain
requirements of the function used. To achieve this, a composition of scaling
and shock functions can be used

\begin{eqnarray*}
R_{t+1} &=&\tilde{f}_{[m,M]}\left( R_{t},R_{s+h},R_{s}\right)  \\
&=&\eta \circ f_{\mathbb{K}}\left( \text{ }\zeta (R_{t}),\text{ }\zeta
(R_{s+h}),\text{ }\zeta (R_{s})\text{ }\right) 
\end{eqnarray*}

\bigskip

with $\zeta :[m,M]\rightarrow \mathbb{K}$, $\eta :\mathbb{K}\rightarrow
\lbrack m,M]$ and $f_{\mathbb{K}}$ a shock function on $\mathbb{K}$ = $%
\mathbb{R}$ or $\mathbb{R}_{+}^{\ast }$.

\bigskip
\bigskip

 Assuming that $\forall $ $%
(R_{t},R_{s+h},R_{s})\in \left[ m,M\right] ^{3}$, $R_{s+h}=R_{s}$, the third statement implies that to be a shock function on $[m,M],$ $\tilde{f}$ must verify : 

\begin{eqnarray*}
R_{t} &=&R_{t+1} \\
&=&\tilde{f}_{[m,M]}\left( R_{t},R_{s+h},R_{s}\right)  \\
&=&\eta \circ \zeta (R_{t})
\end{eqnarray*}

\bigskip

involving that $\zeta $ must\ be an invertible function and $\eta =\zeta ^{-1}$. If this condition is fulfilled, then $\zeta $ and $\zeta ^{-1}$ have the same monotonicity leading to an order-preserving function $\tilde{f}$ on $\left[ m,M\right]^{3}$.



\end{document}
