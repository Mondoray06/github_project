
\documentclass[3pt]{article}
%%%%%%%%%%%%%%%%%%%%%%%%%%%%%%%%%%%%%%%%%%%%%%%%%%%%%%%%%%%%%%%%%%%%%%%%%%%%%%%%%%%%%%%%%%%%%%%%%%%%%%%%%%%%%%%%%%%%%%%%%%%%
\usepackage{amssymb}
\usepackage{amsfonts}
\usepackage{amsmath}
\usepackage{fancyhdr}
\usepackage{geometry}
\usepackage{amscd}

\setcounter{MaxMatrixCols}{10}
%TCIDATA{OutputFilter=LATEX.DLL}
%TCIDATA{Version=4.10.0.2347}
%TCIDATA{Created=Wednesday, March 19, 2008 18:21:16}
%TCIDATA{LastRevised=Monday, August 05, 2019 17:53:54}
%TCIDATA{<META NAME="GraphicsSave" CONTENT="32">}
%TCIDATA{<META NAME="DocumentShell" CONTENT="Articles\SW\JEEP -  A General Purpose Vehicle">}
%TCIDATA{Language=American English}
%TCIDATA{CSTFile=LaTeX article (bright).cst}

\geometry{ hmargin=1.5cm, vmargin=1.5cm }
\newtheorem{theorem}{Theorem}
\newtheorem{acknowledgement}[theorem]{Acknowledgement}
\newtheorem{algorithm}[theorem]{Algorithm}
\newtheorem{axiom}[theorem]{Axiom}
\newtheorem{case}[theorem]{Case}
\newtheorem{claim}[theorem]{Claim}
\newtheorem{conclusion}[theorem]{Conclusion}
\newtheorem{condition}[theorem]{Condition}
\newtheorem{conjecture}[theorem]{Conjecture}
\newtheorem{corollary}[theorem]{Corollary}
\newtheorem{criterion}[theorem]{Criterion}
\newtheorem{definition}[theorem]{D\'{e}finition}
\newtheorem{example}[theorem]{Example}
\newtheorem{exercise}[theorem]{Exercise}
\newtheorem{lemma}[theorem]{Lemme}
\newtheorem{notation}[theorem]{Notation}
\newtheorem{problem}[theorem]{Problem}
\newtheorem{proposition}[theorem]{Proposition}
\newtheorem{remark}[theorem]{Remarque}
\newtheorem{solution}[theorem]{Solution}
\newtheorem{summary}[theorem]{Summary}
\newenvironment{Proof}[1][Proof]{\noindent\textbf{Proof} }{\ \rule{0.5em}{0.5em}}
\input{tcilatex}

\begin{document}

\title{Yield Curve Econometric}
\author{T.\ Monedero \\
%EndAName
Quantitative Research\\
Fixed Income, Natixis\\
}
\maketitle

\begin{abstract}
The aim of this document is to provide some technical elements required for
the Fundamental Review of the Trading Book and more specifically for the
Internal Models Approach of the Yield Curve Econometric in the Historical
Simulation framework.
\end{abstract}

\tableofcontents

.

\bigskip

\bigskip

\bigskip

\section{Introduction}

Defined by the Basel Committee of Banking Supervision, the Fundamental
Review of the Trading Book (FRTB) aims to improve the Basel II.5 regulation
rules and to build a new market risk framework as a response to the
financial crisis. These new standards address a number of both qualitative
and quantitative issues such as\ capital arbitrage between booking and
trading books as well as under-capitalization of the trading book.

\subsection{Standardised Approach}

The Standardised Approach (SA) refers to a set of general risk measurement
techniques proposed in the FRTB, giving a market risk overview of all
bankink institutions. Needed to be calculated and reported to the relevant
supervisor on a monthly basis, this method provide minimum capital
requirements as the sum of three metrics :

\bigskip

\begin{itemize}
\item the sensitivities-based method (SBM)

\item the default risk capital\ (DRC)

\item the residual risk add-on\ (RRAO)
\end{itemize}

\bigskip

Through risk agregation rules, the SBM use the sensitivities of financial
instruments to a predefined risk factor list for each risk classes (Interest
Rate, Credit, Equity, Commodity and Foreign Exchange) to calculate the
delta, vega and curvature risk capital requirements. The DRC is intented to
capture jump-to-default risk that may not be captured by credit spread
shocks under the SBM. To address possible limitations in the SA, the RRAO is
introduced to ensure sufficient coverage of market risks. (trouver des
synonymes et donner plus de details et trouver un moyen \ de glisser limite
de SA avec diversification des classe de risk par exemple dans IMA)

\subsection{Internal Models Approach}

In order to reduce capital requirements, banks can also use the Internal
Models Approach (IMA) for a more accurate measure of their own market risks.
This method, however, entails additional cost of calculation and complexity
and is subject to the supervisory authority agreement at trading desk level\
which must meet two quantitative criteria :

\bigskip

\begin{itemize}
\item P\&L Attribution, a test to determine if P\&L based on risk factors
used by the desks risk management model captures the material drivers of
Actual P\&L

\item Backtesting, a test to determine how well the risks in an internal
model are captured
\end{itemize}

\bigskip

Toujours un module de risque de default plus ou moins similaire decrire les
etapes pla et backtesting =\TEXTsymbol{>} necessit\'{e} hitorique propre

si homologation test avec ES + NMRF pour valider interet de l'approche
standard.

\bigskip

\bigskip Minimum capital requirements p 81

An important part of a bank's trading desk internal risk management model is
the specification

of an appropriate set of market risk factors. Bucketing approach for the RFET

\begin{itemize}
\item definir les risques factors

\item Comment deformer les risk factors (contraintes technique (nb rf) et
fonctionelle (pertinence de la defomation intra et cross rf))

\item Mesure et application des deformations
\end{itemize}

$\ \ \ \ \ $

Most banks should perform quantitative impact studies to compare the two
methods for their trading business.

\section{Yield Curve Econometric}

Yield curve represents the global (majeur le plus important) risk for all
interest rate instruments including derivatives by extension. 

\subsection{Par-Point}

In this approach, all pillars of the yield curve are considered independents
and the econometrics are directly applied to their facial values. Now, the
question is how to compare theses values at different dates. In fact, (
selon la date d'observation et la description des instruments (business days
vs calendar days), on peut avoir d'es instrument differents)

\bigskip

The first way is to extract zero coupon curves from past dates and use them
to value the current yield curve. (ajouter graph)$\times D$

\bigskip

\bigskip

\bigskip

The second way is to consider that rates with sliding maturities such as
money market and swap rates, can be compared regardless the date of
observation. For other rates induced by rolling securities such as Euribor
futures whose maturities are fixed dates, an immediate comparison is
impossible. An alternative can be to compute the current value of theses
rates from the past yield curves using some interpolation rules.

\bigskip 

\section{Historical Simulation Framework}

Historical simulation is a method which assumes that probability law of a
set of variable rely on their past evolution, allowing to capture stylized
facts of financial time series such as fat tails or volatility clustering
without modelling assumptions. For a given time horizion $\alpha $ and a
risk factor $R$ defined on a domain $D$ and\ whose spot level at time $t$ is
noted $R_{t}$, this translates into 

\begin{equation*}
R_{t+1}=f(R_{t},R_{s+h},R_{s}),\text{ \ \ }t-\alpha \leq s<t-h
\end{equation*}

\bigskip 

where function $f$ specifies how to measure past return between $s$ and $s+h$
and how to apply it between $t$ and $t+1.$ Beyond statistical
considerations, $f$ is said wheel defined on $D\times D\times D$ \ if the
following criterias are satisfied:

\bigskip 

\begin{itemize}
\item A total order exist on $D.$

\item The image of $f$ is $D$

\item 
\end{itemize}

\subsection{Absolute returns}

The absolute return is defined as%
\begin{eqnarray*}
R_{t+1} &=&R_{t}+R_{s+h}-R_{s} \\
&=&R_{t}+\delta _{A}
\end{eqnarray*}

\subsection{Relative returns}

\bigskip The reletive return is defined as%
\begin{eqnarray*}
R_{t+1} &=&R_{t}.\left( \frac{R_{s+h}}{R_{s}}\right)  \\
&=&R_{t}+\delta _{R},\text{ \ \ }\delta _{R}=R_{t}.\frac{\left(
R_{s+h}-R_{s}\right) }{R_{s}}
\end{eqnarray*}

\subsection{Absolute-Relative returns}

\bigskip

\subsubsection{Convexe combination}

The most simple way to define an absolute-relative return is to use a
convexe combination between theses two shocks with $\lambda \in \left[ 0,1%
\right] $ 
\begin{equation*}
\delta _{AR}=\lambda .\delta _{A}+(1-\lambda ).\delta _{R}
\end{equation*}

\subsubsection{p-Space}

\bigskip

\bigskip

\bigskip

The Lambert function is the inverse function of 
\begin{equation*}
y=x.\exp (x)\Longleftrightarrow x=W(y)
\end{equation*}

\bigskip using this function, it is possible to define a p-variable $Y_{t}$
by

\bigskip

\begin{equation*}
Y_{t}=p.X_{t}+(1-p).\ln (X_{t})
\end{equation*}

and then%
\begin{equation*}
X_{t}=\left\{ 
\begin{array}{c}
Y_{t},\text{ \ \ \ \ \ \ \ \ \ \ \ \ \ \ \ \ \ \ \ \ \ \ \ \ \ \ \ \ \ \ \ \ 
}p=1 \\ 
\\ 
\exp (Y_{t}),\text{ \ \ \ \ \ \ \ \ \ \ \ \ \ \ \ \ \ \ \ \ \ \ \ \ \ }p=0
\\ 
\\ 
\left( \frac{1-p}{p}\right) .W\left( \frac{p}{1-p}.\exp \left( \frac{Y_{t}}{%
1-p}\right) \right) ,\text{ }p\in \left] 0,1\right[%
\end{array}%
\right.
\end{equation*}

\begin{equation*}
y_{1}^{^{\prime }}=y_{1}+y_{2}-y_{3}
\end{equation*}

and then retrieve $\delta _{AR}$ by%
\begin{equation*}
\delta _{AR}=f^{-1}\left( y_{1}^{^{\prime }}\right) -x_{1}
\end{equation*}

\subsection{Case of bounded risk factors}

\bigskip

\bigskip

\bigskip

\bigskip

\bigskip
https://www.pwc.com/gx/en/advisory-services/basel-iv/basel-iv-revised-standardised-.pdf

https://acpr.banque-france.fr/sites/default/files/medias/documents/eifr%
\_acpr\_presentation\_frtb\_2019.pdf

https://quanteam.fr/2018/07/interview-frtb-bale-iv/

http://finance.sia-partners.com/trading-book-vs-banking-book-la-nouvelle-reglementation

https://www.accenture.com/%
\_acnmedia/pdf-56/accenture-fundamental-review-of-the-trading-book-theory-to-action.pdf

https://www.agefi.fr/asset-management/actualites/video/20190621/natixis-paie-sensibilite-marches-risque-liquidite-277450

https://www.bis.org/bcbs/publ/d457.pdf

https://www.bis.org/bcbs/publ/d352.pdf

https://www.clarusft.com/frtb-internal-models-or-standardised-approach/

https://www.quantifisolutions.com/explaining-the-two-key-frtb-frameworks

https://laithanomics.com/the-fundamental-review-of-the-trading-book-frtb/

https://www.cfainstitute.org/membership/professional-development/refresher-readings/2019/measuring-and-managing-market-risk

https://www.researchgate.net/publication/303719319\_Quelques\_conseils\_pour%
\_ecrire\_un\_article\_scientifique\_en\_anglais

https://www.clarusft.com/fundamental-review-of-the-trading-book-what-you-need-to-know/

C:\TEXTsymbol{\backslash}Users\TEXTsymbol{\backslash}tmonedero\TEXTsymbol{%
\backslash}Downloads\TEXTsymbol{\backslash}ddjp.pdf

\bigskip 

statistical properties to satisfie how to measure past change of risk
factors still remain to define

\end{document}
