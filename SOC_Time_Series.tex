
\documentclass[3pt]{article}
%%%%%%%%%%%%%%%%%%%%%%%%%%%%%%%%%%%%%%%%%%%%%%%%%%%%%%%%%%%%%%%%%%%%%%%%%%%%%%%%%%%%%%%%%%%%%%%%%%%%%%%%%%%%%%%%%%%%%%%%%%%%
\usepackage{amssymb}
\usepackage{amsfonts}
\usepackage{amsmath}
\usepackage{fancyhdr}
\usepackage{geometry}
\usepackage{amscd}

\setcounter{MaxMatrixCols}{10}
%TCIDATA{OutputFilter=LATEX.DLL}
%TCIDATA{Version=4.10.0.2347}
%TCIDATA{Created=Wednesday, March 19, 2008 18:21:16}
%TCIDATA{LastRevised=Friday, May 15, 2020 09:33:58}
%TCIDATA{<META NAME="GraphicsSave" CONTENT="32">}
%TCIDATA{<META NAME="DocumentShell" CONTENT="Articles\SW\JEEP -  A General Purpose Vehicle">}
%TCIDATA{Language=American English}
%TCIDATA{CSTFile=LaTeX article (bright).cst}

\geometry{ hmargin=1.5cm, vmargin=1.5cm }
\newtheorem{theorem}{Theorem}
\newtheorem{acknowledgement}[theorem]{Acknowledgement}
\newtheorem{algorithm}[theorem]{Algorithm}
\newtheorem{axiom}[theorem]{Axiom}
\newtheorem{case}[theorem]{Case}
\newtheorem{claim}[theorem]{Claim}
\newtheorem{conclusion}[theorem]{Conclusion}
\newtheorem{condition}[theorem]{Condition}
\newtheorem{conjecture}[theorem]{Conjecture}
\newtheorem{corollary}[theorem]{Corollary}
\newtheorem{criterion}[theorem]{Criterion}
\newtheorem{definition}[theorem]{D\'{e}finition}
\newtheorem{example}[theorem]{Example}
\newtheorem{exercise}[theorem]{Exercise}
\newtheorem{lemma}[theorem]{Lemme}
\newtheorem{notation}[theorem]{Notation}
\newtheorem{problem}[theorem]{Problem}
\newtheorem{proposition}[theorem]{Proposition}
\newtheorem{remark}[theorem]{Remarque}
\newtheorem{solution}[theorem]{Solution}
\newtheorem{summary}[theorem]{Summary}
\newenvironment{Proof}[1][Proof]{\noindent\textbf{Proof} }{\ \rule{0.5em}{0.5em}}
\input{tcilatex}

\begin{document}

\title{CMS Spread Option}
\author{T.\ Monedero \\
%EndAName
Natixis Fixed Income Department:\\
quantitative analysis }
\maketitle

\begin{abstract}
The aim of this document is to provide some technical elements required for
the reconstruction of Cms Spread Option implied correlation time series
according to today's market conventions.
\end{abstract}

\tableofcontents

.

\bigskip

\bigskip

\bigskip

\bigskip

\bigskip

\bigskip +

\bigskip

\bigskip

\bigskip

\bigskip

\bigskip

\section{Introduction}

\subsection{Notations}

$\ \ \ \ \ D(t,T)$ : Stochastic discount factor at time $t$ for the maturity 
$T$.

\bigskip

$P(t,T)$ : $T$-maturity zero coupon bond value at time $t\leq T$.

\bigskip

$L(T_{R},T_{S,}T_{S}+\tau )$ : Simply compounded Libor rate resetting at
time $T_{R}$ for the start date $T_{S}$ and the tenor $\tau $.

\bigskip

$S(T_{R},T_{S,}T_{E})$ : Swap rate resetting at time $T_{R}$ for the start
date $T_{S}$ and the end date $T_{E}$.

\subsection{Payoff}

A CMS spread option is a financial instrument whose payoff is a function of
the spread between two swap rates of different tenor.

\subsubsection{SingleLook }

For a given maturity $T$ and a strike $K$, a single look CMS spread option
is a one year basis caplet with fixing and payment in arrears 

\bigskip 

\bigskip 

\bigskip 

Its forward premium value at time $t$ is therefore given by%
\begin{eqnarray*}
&& \\
V_{SL}^{SO}(t,T,K) &=&\frac{1}{P^{d}(t,T)}\mathbb{E}_{t}^{Q_{d}}\left[
D^{d}(t,T)\left( S_{T}^{i}-S_{T}^{j}-K\right) _{+}\right] =\mathbb{E}%
_{t}^{Q_{d}^{T}}\left[ \left( S_{T}^{i}-S_{T}^{j}-K\right) _{+}\right]  \\
&&
\end{eqnarray*}

\subsubsection{Cap Floor}

A CMS spread cap is a strip of three month basis caplets with fixing in
advance and in arrears payment.

\bigskip 

\bigskip

Its value at time $t$ is given by%
\begin{eqnarray*}
&& \\
V_{CAP}^{SO}(t) &=&\mathbb{E}_{t}^{Q_{d}}\left[ \tsum\limits_{l=1}^{N}\tau
_{l}D^{d}(t,T_{E}^{l})\left( S_{T_{l}}^{i}-S_{T_{l}}^{j}-K\right) _{+}\right]
=\tsum\limits_{l=1}^{N}\tau _{l}P^{d}(t,T_{E}^{l})\mathbb{E}%
_{t}^{Q_{d}^{T_{E}^{l}}}\left[ \left( S_{T_{l}}^{i}-S_{T_{l}}^{j}-K\right)
_{+}\right]  \\
&&
\end{eqnarray*}%
\bigskip 

\bigskip

\bigskip

\bigskip

\bigskip

\bigskip

\bigskip \bigskip 

\bigskip

\bigskip

\bigskip 

\bigskip \bigskip Market quotes

Both types are quoted by brokers such as Tullet. Both products allow the
investor a view on the shape of the yield curve.\bigskip \bigskip 

and $S_{T}=S(T,T_{S,}T_{E})$ the swap forward rate

\section{CMS Spread Modelling}

\bigskip

\subsection{Convexity Adjusted CMS}

The convexity adjusted constant maturity swap rate is classicaly defined by $%
CMS_{t}\overset{\Delta }{=}\mathbb{E}_{t}^{Q_{d}^{T}}\left[ S_{T}\right] .$
Using the fact that $CMS_{T}=S_{T}$, a single look CMS spread option value
rewrites as 
\begin{eqnarray*}
&& \\
V_{SL}^{SO}(t,T,K) &=&\mathbb{E}_{t}^{Q_{d}^{T}}\left[ \left(
CMS_{T}^{1}-CMS_{T}^{2}-K\right) _{+}\right] 
\end{eqnarray*}

\subsection{Gaussian CMS Spread Model}

Assuming that each convexity adjusted CMS rates is gaussian such that $%
dCMS_{t}^{i}=\sigma _{N}^{i}$ $dW_{t}^{i}$ with $\forall i\neq j,$ $%
\left\langle dW_{t}^{i},dW_{t}^{j}\right\rangle =$ $\rho _{ij}dt$, the $%
\mathcal{F}_{t}$\ conditional law of the convexity adjusted CMS spread $%
X_{T}=CMS_{T}^{i}-CMS_{T}^{j}$ is given by 
\begin{eqnarray*}
&& \\
X_{T}\overset{d}{=}\mathcal{N}\left( X_{t},\sigma _{N}\sqrt{T-t}\right) ,%
\text{ \ \ }\sigma _{N} &=&\sqrt{\left( \sigma _{N}^{i}\right) ^{2}+(\sigma
_{N}^{j})^{2}-2\rho _{ij}\sigma _{N}^{i}\sigma _{N}^{j}} \\
&&
\end{eqnarray*}%
The gaussian model provides a convenient common langage for quoting spread
options and gives an analytical expression of the correlation $\rho _{ij}$
between $CMS^{i}$ and $CMS^{j}$ which is the natural risk factor for this
kind of product. In practice, $\sigma _{N}^{i,j}$ are choosen as atm
swaption implied gaussian volatilities and the convexity adjusted CMS spread 
$X_{t}$ is  obtened by replication. If the CMS spread is assumed to have a
Gaussian distribution, the price of a $T$-maturity caplet spread option \
with a strike $K$ is given by the Bachelier's formula : 

\begin{equation*}
C_{t}^{N}(X_{t},T,K,\sigma _{N\text{ }})=\left( X_{t}-K\right) \text{ }\Phi
\left( \frac{X_{t}-K}{\sigma _{N\text{ }}\sqrt{T-t}}\right) +\sigma _{N\text{
}}\sqrt{T-t}\text{ }\phi \left( \frac{X_{t}-K}{\sigma _{N\text{ }}\sqrt{T-t}}%
\right) 
\end{equation*}

\subsection{Smiled Implied Correlation}

For a given maturity $T,$ a natural cubic correlation spline $\rho (T,K_{i})$%
\ is used to fit\ a set of CMS Spread\ option market prices $\left(
V_{SL}^{SO\ast }\right) _{i=1..n}$, i.e such that 
\begin{eqnarray*}
\forall i &=&1...n,\text{ \ }V_{SL}^{SO\ast
}(t,T,K_{i})=C_{t}^{N}(X_{t},T,K_{i},\sigma _{N\text{ }}(K_{i},T)) \\
&&
\end{eqnarray*}%
The CMS Spread distribution is then projected on a dynamic range - wrt the
money - by building another relative correlation spline for a set of
predefined moneyness strike $K_{j}^{m}\in \left[ K_{\min }^{m},K_{\max }^{m}%
\right] $ : 
\begin{eqnarray*}
\rho _{Atm}^{T} &=&\rho (T,K=X_{t}) \\
\rho _{m}(T,K_{j}^{m}) &=&\rho (T,X_{t}+K_{j}^{m})-\rho _{Atm}^{T}
\end{eqnarray*}

\subsection{Cap Floor Pricing}

\subsubsection{Paylag Adjustment}

\bigskip 

\subsubsection{Term Structure Interpolation}

\section{CMS Spread Correlation Time series}

For Butterfly Arbitrage, maturity dependency can be omitted

\bigskip 

\begin{itemize}
\item fit quickly

\item compliant with the cubic spline

\item compliant with the cubic splineproduce butterfly arbitrage free prices
\end{itemize}

\subsection{Butterfly Arbitrage condition}

\bigskip 

Setting $\xi (K)=\frac{X_{t}-K}{\sigma _{N}(T,K)\sqrt{T-t}},$ Bachelier's
formula rewrittes as%
\begin{eqnarray*}
&& \\
C_{t}^{N}(X_{t},T,K,\sigma _{N\text{ }}(K,T)) &=&\sigma _{N\text{ }}(K,T)%
\sqrt{T-t}\left( \phi (\xi )+\xi \Phi (\xi )\right)  \\
&&
\end{eqnarray*}%
Assuming $\sigma _{N\text{ }}:\mathbb{R\rightarrow R}_{+}$ and\ $\sigma _{N%
\text{ }}\in \mathcal{C}^{2}\mathbb{(}\mathbb{R})\ $CMS spread Cdf and Pdf
can be deduced as follow :

\begin{eqnarray*}
\frac{\partial C_{t}^{N}}{\partial K} &=&\frac{\sigma _{N\text{ }}^{^{\prime
}}(K)}{\sigma _{N\text{ }}(K)}C_{t}^{N}+\sigma _{N\text{ }}(K)\Phi (\xi )%
\frac{\partial \xi }{\partial K} \\
&& \\
\frac{\partial ^{2}C_{t}^{N}}{\partial K^{2}} &=&2\sigma _{N\text{ }%
}^{^{\prime }}(K)\Phi (\xi )\frac{\partial \xi }{\partial K}+\sigma _{N\text{
}}^{^{\prime \prime }}(K)\left( \phi (\xi )+\xi \Phi (\xi )\right) +\sigma
_{N\text{ }}(K)\phi (\xi )\left( \frac{\partial \xi }{\partial K}\right)
^{2}+\sigma _{N\text{ }}(K)\Phi (\xi )\frac{\partial ^{2}\xi }{\partial K^{2}%
} \\
&&
\end{eqnarray*}

Computing first and second derivatives of $\xi $

\begin{eqnarray*}
\frac{\partial \xi }{\partial K} &=&-\frac{1}{\sigma _{N\text{ }}(K)}\left(
1+\sigma _{N\text{ }}^{^{\prime }}(K)\xi \right)  \\
\frac{\partial ^{2}\xi }{\partial K^{2}} &=&-\frac{1}{\sigma _{N\text{ }}(K)}%
\left( \sigma _{N\text{ }}^{^{\prime \prime }}(K)\xi +2\sigma _{N\text{ }%
}^{^{\prime }}(K)\frac{\partial \xi }{\partial K}\right)  \\
&&
\end{eqnarray*}%
leads to the equivalence reliatioship between the price convexity and the
gaussian volatility convexity for the butterfly arbitrage condition

\begin{eqnarray*}
\frac{\partial ^{2}C_{t}^{N}}{\partial K^{2}} &=&\phi (\xi )\left( \sigma _{N%
\text{ }}^{^{\prime \prime }}(K)+\frac{\left( 1+\sigma _{N\text{ }%
}^{^{\prime }}(K)\xi \right) ^{2}}{\sigma _{N\text{ }}(kK)}\right) \text{ \
\ and \ \ }\frac{\partial ^{2}C_{B}}{\partial K^{2}}>0\Leftrightarrow \sigma
_{N\text{ }}^{^{\prime \prime }}(K)>0 \\
&&
\end{eqnarray*}

\subsection{Implied Correlation Parametrization}

As seen before, the relationship between gaussian CMS spread implied
volatility and gaussian CMS implied correlation is of the following form

\begin{equation*}
\sigma _{N}(K)=\sqrt{\alpha ^{2}+\beta ^{2}-2\alpha \beta \rho (K)},\text{ \
\ (}\alpha ,\beta )\in (\mathbb{R}_{+})^{2}
\end{equation*}%
then%
\begin{eqnarray*}
\sigma _{N}(K) &>&0\Leftrightarrow \rho (K)<1+\frac{\left( \alpha -\beta
\right) ^{2}}{2\alpha \beta }\text{ }\forall K \\
&&
\end{eqnarray*}
that is always true since $\forall K,$ $\left\vert \rho (K)\right\vert <1$.
The Butterfly arbitrage condition in terms of implied correlation is
adressed through the following equations

\QTP{Body Math}
\begin{eqnarray*}
\sigma _{N}^{^{\prime }}(K) &=&-\alpha \beta \frac{\rho ^{^{\prime }}(K)}{%
\sigma _{N}(K)} \\
\sigma _{N}^{^{\prime \prime }}(K) &=&-\frac{\alpha \beta }{\sigma
_{N}^{2}(K)}\left( \rho ^{^{^{\prime \prime }}}(K)\sigma _{N}(K)+\alpha
\beta \frac{\left( \rho ^{^{\prime }}(K)\right) ^{2}}{\sigma _{N}(K)}\right) 
\\
&&
\end{eqnarray*}%
Obviously $\rho \in \mathcal{C}^{2}(\mathbb{R})$ implies that $\sigma
_{N}\in \mathcal{C}^{2}(\mathbb{R})$ and the volatility convexity is
equivalent to 
\begin{eqnarray*}
&& \\
\forall K,\text{ \ \ }\rho ^{^{^{\prime \prime }}}(K)\sigma
_{N}(K)^{2}+\alpha \beta \rho ^{^{\prime }}(K)^{2} &<&0 \\
&&
\end{eqnarray*}%
Given a parabolic parametrization of the correlation $\rho
(K)=aK^{2}+bK+\rho _{atm},$ this condition rewrittes as%
\begin{eqnarray*}
&& \\
2a\left( \alpha ^{2}+\beta ^{2}-2\alpha \beta \rho _{atm}\right) +\alpha
\beta b^{2} &<&0 \\
&&
\end{eqnarray*}

\QTP{Body Math}
that is strike independante and equivelent to $\rho _{\max }=\left( \rho
_{atm}-\frac{b^{2}}{4a}\right) <\frac{\alpha ^{2}+\beta ^{2}}{2\alpha \beta }%
.$

\QTP{Body Math}
$\bigskip $

\begin{remark}
As $\left\vert \rho _{atm}\right\vert \leq 1,$ $\alpha ^{2}+\beta
^{2}-2\alpha \beta \rho _{atm}\geq \alpha ^{2}+\beta ^{2}-2\alpha \beta
=(\alpha -\beta )^{2}>0$ then inevitably $a<0.$
\end{remark}

\end{document}
